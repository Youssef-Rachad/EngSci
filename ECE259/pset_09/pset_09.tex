\documentclass[answers]{exam}
\usepackage{../../template}
\author{niceguy}
\title{Problem Set 9}
\begin{document}
\maketitle

\begin{questions}

\question{Consider an arbitrary distribution of volume currents in a vacuum
and the magnetic flux density vector due to these currents at an arbitrary point in space. If the magnitude of the current density vector is doubled everywhere, the magnetic flux density vector consider}

\begin{solution}
    A, becomes twice large in magnitude and keeps the same direction. This is because from the Biot-Savart Law, $\vec B$ is proportional to $I$.
\end{solution}

\question{A rectangular wire loop of edge lengths $a$ and $b$ in air carries a steady current of intensity $I$ ($I > 0$), as shown in Fig. Q4.3. The magnetic flux density vector $\vec B$ at the point $M$ in the figure can be represented as}

\begin{solution}
    C, $\vec B = B_z\hat a_z, B_z > 0$.
\end{solution}

\question{Three identical solenoidal coils, wound uniformly and densely with $N$ turns of thin wire, are positioned in space as shown in Fig. Q4.4. The axes of coils lie in the same plane and the permeability everywhere is $\mu_o$. Let $I_1, I2$ and $I_3$ denote the intensities of time-invariant currents in the coils. Consider the following two cases: (a) $I_1 = I_2 = I_3 = I$ and (b) $I_1 = I, I_2 = I3 = 0$. If $I > 0$, the magnetic flux density at the center of the system (the point $P$) for case (a) is}

\begin{solution}
    C, smaller than the magnetic flux density at the same point for case (b). In case (a), $\vec B$ cancels out to 0 by symmetry, while $\vec B$ is nonzero in case (b).
\end{solution}

\question{A time-invariant current of intensity $I( > 0)$ is established in a cylindrical copper conductor. The conductor is situated in air. The circulation (line integral) of the magnetic flux density vector, $\vec B$, along a contour $C$ composed from two circular and two radial parts and positioned outside the conductor, as shown in Fig. Q4.5, is}

\begin{solution}
    E, zero. The small radial components of the contour do not matter, as they are perpendicular to $\vec B$, which vanishes. In the other cases, $\vec B$ is either parallel or anti parallel to $d\vec l$, so the integral becomes $\mu_0I-\mu_0I = 0$.
\end{solution}

\question{Consider two very long metallic conductors, one of a circular and the other of square cross section. Both conductors carry steady currents of the same density. If the same circular contour $C$ is positioned inside each of the conductor, as in Fig. Q4.6, the circulation of the magnetic flux density vector along $C$ in the circular conductor is}

\begin{solution}
    B, the same as. This is because the enclosed currents are the same.
\end{solution}

\question{Consider two identical cylindrical metallic conductors carrying steady currents of the same intensity and Amperian contours positioned around each of them. The first contour is circular, while the other one has a square shape, as shown in Fig. Q4.7. The circulation of the magnetic flux density vector along the circular contour is}

\begin{solution}
    B, the same as. Same reason as above.
\end{solution}

\question{ A contour composed of eight straight segments is positioned in air near a very long wire conductor with a steady current of intensity $I$(Fig. Q4.8). The line integral of the magnetic flux density vector due to this current along the part of the contour between points $M$ and $Q$, via $N$ and $P$, equals}

\begin{solution}
    B, $\frac{\mu_0I}{2}$. Consider mirroring the circuit along the vertical axis. Then the total path integral is $\mu_0I$. Then by symmetry, it is half of that.
\end{solution}

\question{The inner and outer conductors of a coaxial cable carry steady currents of the same intensity and opposite directions. The cable dielectric and conductors are nonmagnetic, and the surrounding medium is air. If the current intensity in the inner conductor is increased, while keeping the current in the outer conductor unchanged, the magnetic flux density at every point of the outer conductor}

\begin{solution}
    A, increases. The closed path integral of $\vec B$ is equal to the enclosed current. Since enclosed current increases, $\vec B$ increases.
\end{solution}

\question{For the coaxial cable from the previous question, assume that the current intensity in the outer conductor is decreased, while keeping the current in the inner conductor unchanged. As a result, the magnetic flux density at every point of the inner conductor (not considering the points at the conductor axis)}

\begin{solution}
    C, remains the same. Same logic as above, but enclosed current remains the same.
\end{solution}

\question{An insulated metallic strip folded as in Fig. Q4.9 carries a steady current of intensity $I$. The width of the strip is $a=20d$, where $d$ is the diameter of the cylindrical cavity formed by the strip. With this, the magnitude of the $\vec B$ field at the point $P$ inside the cavity (see the figure) is}

\begin{solution}
    No asterisk no do
\end{solution}

\question{Consider the folded strip conductor from the previous question, and assume that $a$ is made twice larger, while keeping $I$ and $d$ the same. The magnitude of the $\vec B$ field at the point $P$ in the modified structure is}

\begin{solution}
    no asterisk
\end{solution}

\question{Two very long and wide conducting strips are placed in air parallel and very close to each other. Steady currents of the same intensity, $I$, and opposite directions flow through the strips, as shown in Fig. Q4.10. If the current direction is reversed in one of the strips, the magnetic field between the strip, away from the strip edges}

\begin{solution}
    B, becomes noticeably weaker. Approximate the long and wide conducting strips as wires (physics moment).
\end{solution}

\question{Consider the field pattern (showing lines of a vector field a in a part of free space) in Fig. Q4.11(a) and that in Fig. Q4.11(b), and whether each of the fields is divergence-free ($\vec \nabla \cdot \vec a = 0$) and/or curl-free ($\vec \nabla \vec a = 0$). Which of the following statements is true?}

\begin{solution}
    A, Field in Fig. Q4.11(a) is divergence-free and field in Fig. Q4.11(b) is curl-free. THis is because in (a), $\vec a = f(y)\hat a_x$, and in (b), $\vec a = f(x)\hat a_x$.
\end{solution}

\question{In a certain region in free space, there is a uniform magnetic field, with flux density $\vec{B_0}$. The volume current density in that region}

\begin{solution}
    B, is zero. This is because $\oint_ctrclockwise \vec B \cdot d\vec l = \mu I$, but the first integral is always zero for a uniform field.
\end{solution}

\question{A sphere of radius $a$ is placed in free space near a very long, straight wire carrying a steady current of intensity $I$. The distance of the sphere center from the axis is $d$. The magnetic flux through the sphere surface depends on}

\begin{solution}
    E, non of the above parameters. This is because $\oiint \vec B \cdot d\vec S = 0$.
\end{solution}

\question{Consider an imaginary open conical surface in a uniform steady magnetic field of flux density $\vec B$ = 1 T. The height (length) of the cone is $h = 20$ cm and the radius of its opening is $a = 10$ cm. The vector $\vec B$ makes an angle $\alpha = 45$ with the cone axis as in Fig.Q4.12. If $h$ is doubled (without changing $a$, $B$, and $\alpha$), the magnetic flux through the conical surface (oriented downward)}

\begin{solution}
    C.
\end{solution}

\question{Consider a vertical cylinder in a steady magnetic field in free space. If $\Phi_1$ denotes the magnetic flux through the lower basis of the cylinder and $\Phi_2$ that through the upper basis with both surfaces oriented in the same way (upwards), we have that}

\begin{solution}
\end{solution}

\question{If both curl and divergence of a vector $\vec a$ at a point are zero, then $\vec a$ must be zero at that point.}

\begin{solution}
    Obviously no. Consider $\vec a = k \hat a_x$.
\end{solution}

\question{If a vector $\vec a$ at a point is zero, then both the curl and divergence of $\vec a$ must be zero at that point.}

\begin{solution}
    Obviously no. Consider $\vec a = (x+y)\hat a_x$.
\end{solution}

\question{An infinintely long cylindrical bar magnet of radius $a$, is permanently magnetized with a uniform magnetization, and the magnetization vector, of magnitude $M$, is parallel to the bar axis. Magnitudes of the magnetization volume and surface current density vectors, $\vec{J_m}$ and $\vec{J_{ms}}$, over the volume and surface of the magnet, respectively, are}

\begin{solution}
    B, $J_m = 0, J_{ms} = M$. This is because $J_m = \vec\nabla \times \vec M = 0$, as $\vec M$ is a constant. Now $J_{ms} = \hat a_n \times \vec M$, which is nonzero, as the surface normal is perpendicular to the bar axis.
\end{solution}

\question{Using the magnetization vector, $\vec M$, of the bar magnet from the previous question, the magnetic flux density vector inside the magnet can be expressed as}

\begin{solution}
    D, $\vec B = \mu_0 (\vec H + \vec M) = \mu_0 \vec M$, since $\vec H$ vanishes in the centre of a magnet.
\end{solution}

\question{The magnetization vector in a ferromagnetic body is given by $\vec M = M(x)\hat a_y$. The magnetization volume current density vector inside this body can be represented as}

\begin{solution}
    C, $\vec J_m = J_m(x)\hat a_z$. This is because $\vec J_m = \vec\nabla \times \vec M$, so $J_m$ points in $\hat a_z$ and is a function of $x$.
\end{solution}

\question{Assuming that the magnetic flux density vector $\vec B$ inside the bar magnet from Questions 5.1 and 5.2 is known, the associated magnetic field intensity vector (in the magnet) equals}

\begin{solution}
    E. $\vec H$ vanishes in the centre of a magnet.
\end{solution}

\question{Determine the magnetic flux density at a point on the axis of a solenoid with radius $b$ and length $L$, and with a current $I$ in its $N$ turns of closely wound coil. Show that the result reduces to the following equation as $L \rightarrow \infty$}

\begin{solution}
    For an infinitely long solenoid, we can always pick a coordinate system where the point is the origin. Then
    \begin{align*}
        d\vec B &= \frac{\mu Idz}{4\pi} \frac{N}{L} \oint_C \frac{bd\phi\hat a_\phi \times (-b\hat a_r-z\hat a_z)}{(b^2+z^2)^{\frac{3}{2}}} \\
                &= \frac{\mu Idz}{4\pi} \frac{N}{L} \int_0^{2\pi} \frac{b(b\hat a_z - z\hat a_r)d\phi}{(b^2+z^2)^{\frac{3}{2}}} \\
                &= \frac{\mu Idz}{4\pi} \frac{N}{L} \frac{2\pi b^2}{(b^2+z^2)^{\frac{3}{2}}} \hat a_z \\
                &= \frac{\mu Ib^2}{2} \frac{N}{L} \frac{dz}{(b^2+z^2)^{\frac{3}{2}}} \hat a_z
    \end{align*}
    where the $\hat a_r$ term in the integral cancels itself out. Then total $\vec B$ is
    \begin{align*}
        \vec B &= \int_{-\frac{L}{2}}^{\frac{L}{2}} \frac{\mu Ib^2}{2} \frac{N}{L} \frac{dz}{(b^2+z^2)^{\frac{3}{2}}} \hat a_z \\
               &= \frac{\mu Ib^2}{2} \frac{N}{L} \frac{1}{b^2} \frac{2L}{\sqrt{L^2+4b^2}} \\
               &= \frac{\mu IN}{\sqrt{L^2+4b^2}}
    \end{align*}
    Then as $L\rightarrow\infty$, we have
    $$\vec B = \frac{\mu_0 IN}{L}$$
\end{solution}

\question{A very long, thin conducting strip of width $d$ lies in the $xz$-plane between $x=\pm\frac{w}{2}$. A surface current $\vec J_s = J_{s0}\hat a_z$ flows in the strip. Find the magnetic flux density at an arbitrary point outside the strip.}

\begin{solution}
    For an infinitely long wire,
    $$\vec B = \frac{\mu_0 I}{2\pi r}$$
    Then the flux for a strip of width $dx'$ is
    $$d\vec B = \frac{\mu_0 J_{s0}dx'}{2\pi r} \left(-\frac{y}{r} \hat a_x + \frac{x-x'}{r} \hat a_y\right) = \frac{\mu_0 J_{s0}dx'}{2\pi} \left(-\frac{y}{(x-x')^2+y^2} \hat a_x + \frac{x-x'}{(x-x')^2+y^2} \hat a_y\right)$$
    Integrating,
    \begin{align*}
        \vec B &= \int_{-\frac{w}{2}}^{\frac{w}{2}} d\vec B \\
               &= \frac{\mu_i J_{s0}}{2\pi}\left(\left(\arctan\left(\frac{2x-w}{2y}\right) - \arctan\left(\frac{2x+w}{2y}\right)\right)\hat a_x + \left(\frac{1}{2}\ln\frac{(2x+w)^2+4y^2}{(2x-w)^2+4y^2}\right)\hat a_y\right)
    \end{align*}
    where some steps were skipped.
\end{solution}

\question{An infinitely long straight conductor with a circular cross section of radius $b$ carries a steady current $I$. Determine the magnetic flux density inside and outside the conductor.}

\begin{solution}
    Note that due to symmetry, $\vec B$ only has an $\hat a_\phi$ component. Inside the conductor,
    \begin{align*}
        \oint_C \vec B \cdot d\vec l &= \mu I_{\text{enc}} \\
        2\pi rB_\phi &= \mu I \frac{r^2}{b^2} \\
        B_\phi &= \frac{\mu I r}{2\pi b^2} \\
        \vec B &= \frac{\mu Ir}{2\pi b^2} \hat a_\phi
    \end{align*}
    Outside,
    \begin{align*}
        \oint_C \vec B \cdot d\vec l &= \mu I_{\text{enc}} \\
        2\pi r B_\phi &= \mu I \\
        B_\phi &= \frac{\mu I}{2\pi r} \\
        \vec B &= \frac{\mu I}{2\pi r} \vec \hat a_\phi
    \end{align*}
\end{solution}

\question{A cylindrical bar magnet of radius $b$ and length $L$ has a uniform magnetization $\vec M = M_0\hat a_z$ along its axis. Determine the magnetic flux density at an arbitrary distant point.}

\begin{solution}
    No asterisk
\end{solution}

\question{Find the total magnetic flux through a circular toroid with a rectangular cross section of height $h$. The inner and outer radii of the toroid are $a$ and $b$, respectively. A current $I$ flows in $N$ turns of closely wound wire around the toroid. Determine the percentage of error if the flux is found by  multiplying the cross-sectional area by the flux density at the mean radius.}

\begin{solution}
    Note that by symemtry, $\vec B$ only has a $\hat a_\phi$ component.
    \begin{align*}
        \int \vec B \cdot d\vec l &= \mu I_{\text{enc}} \\
        2\pi rB_\phi &= \mu NI \\
        B_\phi &= \frac{\mu NI}{2\pi r} \\
        \vec B &= \frac{\mu NI}{2\pi r} \hat a_\phi
    \end{align*}
    Then the flux is given by
    \begin{align*}
        \Phi &= \int_S \vec B \cdot d\vec S \\
             &= h \int_a^b \frac{\mu NI}{2\pi r} dr \\
             &= \frac{\mu NIh}{2\pi} \ln\frac{b}{a}
    \end{align*}
    If we take the average $\vec B$, then
    $$\Phi = (b-a)h \times \frac{\mu NI}{\pi(b+a)}$$
    Percentage error is
    $$\left(\frac{2(b-a)}{(b+a)\ln\frac{b}{a}} - 1\right) \times 100\%$$
\end{solution}

\question{A current $I$ flows lengthwise in a very long, thin conducting sheet of width $w$, as shown below. Assuming that the current flows into the paper, determine the magnetic flux density $\vec B$ at points $P_1(0, d)$ and $P_2(2w/3, d)$.}

\begin{solution}
    For $P_1$,
    \begin{align*}
        d\vec B &= \frac{\mu Idx}{4\pi w} \int_{-\infty}^\infty \frac{-\hat a_zdz \times (d\hat a_y - x\hat a_x - z\hat a_z)}{(d^2+x^2+z^2)^{\frac{3}{2}}} \\
                &= \frac{\mu Idx}{4\pi w} \int_{-\infty}^\infty \frac{d\hat a_x x\hat a_y}{(d^2+x^2+z^2)^{\frac{3}{2}}} \\
                &= \frac{\mu Idx}{4\pi w} \times (d\hat a_x + x\hat a_y) \times \frac{2}{x^2+d^2} \\
                &= \frac{\mu I(d\hat a_x + x\hat a_y) dx}{2\pi w(x^2+d^2)} \\
        \vec B &= \int_0^w \frac{\mu I(d\hat a_x + x\hat a_y) dx}{2\pi w(x^2+d^2)} \\
               &= \frac{\mu I}{2\pi w} \left(\arctan\frac{w}{d} \hat a_x + \frac{1}{2} \ln\frac{w^2+d^2}{d^2}\right)
    \end{align*}
    Using superposition, $\vec B$ for $P_2$ can be split into the contributions from the left and right component. When mirroring, the $\hat a_y$ component flips its sign, while the $\hat a_x$ component doesn't. Hence
    $$\vec B_R = \frac{3\mu I}{2\pi w} \left(\arctan\frac{2w}{3d} \hat a_x + \frac{1}{2} \ln\frac{w^2+9d^2}{9d^2}\right)$$
    and
    $$ \vec B_L = \frac{3\mu I}{4\pi w} \left(\arctan\frac{2w}{3d} \hat a_x - \frac{1}{2} \ln\frac{4w^2+9d^2}{9d^2}\right)$$
    Then
    $$\vec B = \vec B_R + \vec B_L$$
\end{solution}

\question{A current $I$ flows in a $w\times w$ square loop as shown in Fig. a) below. Find the magnetic flux density
at the off-center point $P (w/4, w/2$).}

\begin{solution}
    From Fig .b),
    \begin{align*}
        \vec B &= \frac{\mu I}{4\pi} \int_0^w \frac{dy\hat a_y \times (b\hat a_x + a\hat a_y - y\hat a_y)}{(b^2 + (y-a)^2)^{\frac{3}{2}}} \\
               &= \frac{\mu I}{4\pi} \int_0^w \frac{-bdy\hat a_z}{(b^2 + (y-a)^2)^{\frac{3}{2}}} \\
               &= -\frac{\mu I}{4\pi} \frac{1}{b}\left(\frac{w-a}{\sqrt{b^2+(w-a)^2}} + \frac{a}{\sqrt{a^2+b^2}}\right) \\
               &= -\frac{\mu I}{4\pi b}\left(\frac{w-a}{\sqrt{b^2+(w-a)^2}} + \frac{a}{\sqrt{a^2+b^2}}\right)
    \end{align*}
    Then summing all 4 components,
    \begin{align*}
        \vec B &= -\frac{\mu I}{\pi w}\frac{4}{\sqrt{5}} -\frac{\mu I}{2\pi w}\left(\frac{3}{\sqrt{13}} + \frac{1}{\sqrt{5}}\right) - \frac{\mu I}{3\pi w}\frac{4}{\sqrt{13}} - \frac{\mu I}{2\pi w} \left(\frac{1}{\sqrt{5}} + \frac{3}{\sqrt{13}}\right) \\
               &= \frac{\mu I}{3\pi w}(\sqrt{13} + 3\sqrt{5})
    \end{align*}
\end{solution}

\question{A long wire carrying a current $I$ folds back with a semicircular bend of radius $b$ as in Fig. Q6-38. Determine the magnetic flux density $\vec B$ at the center point $P$ of the bend.}

\begin{solution}
    By symmetry, contribution to $\vec B$ due to the straight wires is parallel to $\hat a_z$. Now assume each straight wire extends to the left also. We know the contribution from each wire is the
    $$\vec B = \frac{\mu I}{2\pi b}\hat a_z$$
    By symmetry, the left and right halves of this extended wire have the same contribution, hence the right half contributes
    $$\vec B = \frac{\mu I}{4\pi b} \hat a_z$$
    Now the contribution from the top wire is the same by symmetry, so
    $$\vec B = \frac{\mu I}{2\pi b} \hat a_z$$
    Contribution due to the semicircle is
    \begin{align*}
        \vec B &= \frac{\mu I}{4\pi} \int_{\frac{\pi}{2}}^{\frac{3\pi}{2}} \frac{bd\phi\hat a_\phi \times (-b\hat a_r)}{b^3} \\
               &= \frac{\mu I}{4\pi b} \int_{\frac{\pi}{2}}^{\frac{3\pi}{2}} d\phi \hat a_z \\
               &= \frac{\mu I}{4b} \hat a_z
    \end{align*}
    Then
    $$\vec B_{\text{total}} = \frac{\mu I}{2b}\left(\frac{1}{\pi} + \frac{1}{2}\right)\hat a_z$$
\end{solution}

\question{Two identical coaxial coils, each of $N$ turns and radius $b$, are separated by a distance $d$, as depicted in Fig. Q6-39. A current $I$ flows in each coil in the same direction. Find the magnetic flux density midway between the coils, and show that $\frac{dB_x}{dx}$ vanishes at the midpoint. Find the relation between $b$ and $d$ such that $\frac{d^B_x}{dx^2}$ also vanishes at the midpoint. Such a pair of coils are used to obtain an approximately uniform magnetic field in the midpoint region. They are known as Helmholtz coils.}

\begin{solution}
    Nonessential.
\end{solution}

\question{A circular rod of magnetic material with permeability $\mu$ is inserted coaxially in the long solenoid of Fig. 6-4. The radius of the rod, $a$, is less than the inner radius, $b$, of the solenoid. The solenoid has $n$ turns per unit length and carries a current $I$. Find $\vec H$, $\vec B$, and Minside the solenoid, as well as current densities $\vec J_m$ and $\vec J_{m,s}$.}

\begin{solution}
    \begin{align*}
        \oint \vec H \cdot d\vec l &= I \\
        Hl &= nlI \\
        H &= nI
    \end{align*}
    Where $\vec H$ points in the $-\hat a_z$ direction. The magnetic flux density is then
    $$\vec B = \mu \vec H = \begin{cases} -\mu nI\hat a_z & r \leq a \\ -\mu_0 nI\hat a_z & a < r < b\end{cases}$$
    Now
    $$\vec M = \chi_m\vec H = \begin{cases} -\left(\frac{\mu}{\mu_0}-1\right)nI\hat a_z & r \leq a \\ 0 & a < r < b\end{cases}$$
    Finally
    $$\vec J_m = \vec\nabla \times \vec M = 0$$
    since $\vec M$ is a constant.
    $$\vec J_{m,s} = \vec M \times \hat a_n = \vec M \times \hat a_r = -\left(\frac{\mu}{\mu_0}-1\right)nI\hat a_\phi$$
    Note that there is no surface current on the top and bottom surfaces, since $\vec M$ and $\hat a_z$ are parallel.
\end{solution}

\question{Figure 6-37 shows an infinitely long solenoid with air core having a radius $b$ and $n$ closely wound turns per unit length. The windings are slanted at an angle $\alpha$ and carry a current $I$. Determine the magnetic flux density both inside and outside the solenoid.}

\begin{solution}
    Nonessential.
\end{solution}

\question{Determine the magnetic flux density inside an infinitely long solenoid in free space having $n$ closely wounds turns per unit length and carrying a current $I$.}

\begin{solution}
    Using Amp\`ere's Law,
    \begin{align*}
        \oint \vec H \cdot d\vec l &= nlI \\
        Hl &= nlI \\
        H &= nI \\
        B &= \mu_0 nI
    \end{align*}
\end{solution}

\question{A current $I$ flows in the inner conductor of an infinitely long coaxial line and returns via the outer conductor. The radius of the inner conductor is $a$, and the inner and outer radius of the outer conductor are $b$ and $c$, respectively. Find the magnetic flux density $\vec B$ and plot is as a function of $r$ for $0 < r < c$.}

\begin{solution}
    Nonessential.
\end{solution}

\question{A ferromagnetic sphere of radius $b$ is magnetized uniformly with magnetization $\vec M = \hat a_z M_0$. Find the current densities $\vec J_m$, $\vec J_{ms}$, and flux density $\vec B$.}

\begin{solution}
    $$\vec J_m = \vec\nabla \times \vec M = 0$$
    since $\vec M$ is a constant. Now
    $$\vec J_{m,s} = \vec M \times \hat a_n = M_0\hat a_z \times \hat a_R = M_0\hat a_z \times (\sin\theta\hat a_r + \cos\theta \hat a_z) = M_0\sin\theta \hat a_\phi$$
    By symmetry, $\vec B$ only has an $\hat a_z$ component. Now
    \begin{align*}
        \vec B &= \int_0^\pi \frac{\mu J_{m,s}bd\theta}{4\pi} \int_0^{2\pi} \frac{b\sin\theta d\phi\hat a_\phi \times (-b\sin\theta\hat a_r -b\cos\theta\hat a_z)}{b^3} \\
               &= \frac{\mu M_0}{4\pi} \int_0^\pi \sin\theta d\theta \int_0^{2\pi} \sin\theta(\sin\theta\hat a_z - \cos\theta\hat a_r)d\phi \\
               &= \frac{\mu M_0}{4\pi} \int_0^\pi \sin\theta d\theta \times 2\pi\sin^2\theta \hat a_z \\
               &= \frac{\mu M_0}{2} \int_0^\pi \sin^3\theta d\theta \hat a_z \\
               &= \frac{\mu M_0}{2} \int_0^\pi (\cos^2\theta-1)d(\cos\theta) \hat a_z \\
               &= \frac{\mu M_0}{2} \frac{4}{3} \hat a_z \\
               &= \frac{2\mu M_0}{3} \hat a_z
    \end{align*}
\end{solution}

\question{In certain experiments it is desirable to have a region of constant magnetic flux density. This can be created in an off-center cylindrical cavity that is cut in a very long cylindrical conductor carrying a uniform current density. Refer to the cross section in figure below. The uniform axial current density is $\vec J = \hat a_zJ$. Find the magnitude and direction of $\vec B$ in the cylindrical cavity whose axis is displaced from that of the conducting part by a distance $d$.}

\begin{solution}
    Without the cavity, the flux density inside the cylinder is
    $$\vec B_1 = \frac{\mu Jr}{2} \hat a_\phi = \frac{\mu Jr}{2} (-\sin\phi\hat a_x + \cos\phi\hat a_y) = \frac{\mu Jr}{2} \left(-\frac{y}{r}\hat a_x + \frac{x}{r}\hat a_y\right) = \frac{mu J}{2} (-y\hat a_x + x\hat a_y)$$
    Similarly, using a shifted origin,
    $$\vec B_2 = -\frac{\mu Jr'}{2} \hat a_{\phi'} = \frac{mu J}{2} (y'\hat a_x - x'\hat a_y) = \frac{\mu J}{2} (y\hat a_x - (x+d)\hat a_y)$$
    Now total flux density is
    $$\vec B_1 + \vec B_2 = \frac{\mu J}{2} (-y\hat a_x + x\hat a_y + y\hat a_x - (x-d)\hat a_y) = \frac{\mu J}{2} d\hat a_y$$
\end{solution}

\end{questions}
\end{document}
