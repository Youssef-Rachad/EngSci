\documentclass[12pt]{article}
\usepackage{../../template}
\author{niceguy}
\title{Lecture 29}
\begin{document}
\maketitle

\section{Magnetic Flux}

\begin{defn}
    Magnetic Flux is defined as
    $$\Phi = \iint_S \vec B \cdot d\vec S$$
    with units of Wb.
\end{defn}

Since
$$\vec B = \vec \nabla \times \vec A$$
we can transform the above to
$$\Phi = \ointctrclockwise_C \vec A \cdot d\vec l = \iint_S \vec \nabla \times \vec A \cdot d\vec S$$

\begin{ex}
    Find the magnetic flux within a toriod with a gap. \\
    Now $\vec B$ field along the toroid is constant. Comparing $\mu$, this gives $H_{\text{gap}} >> H_{\text{core}}$. From Amp\`ere's Law,
    $$\oint_C \vec H \cdot d\vec l = H_{\text{core}}L_{\text{core}} + H_{\text{gap}}L_{\text{gap}} = NI_0$$
    Isolating for $B_0$,
    $$B_0 = \frac{NI_0}{\frac{L_c}{\mu_0\mu_r} + \frac{L_g}{\mu_0}}$$
    The flux is then
    $$\Phi = \iint_S \vec B \cdot d\vec{S} \approx B_0S$$
\end{ex}

\end{document}
