\documentclass[12pt]{article}
\usepackage{../../template}
\author{niceguy}
\title{Lecture 4}
\begin{document}
\maketitle

\section{Electric Fields}

\begin{ex}
	Electric Field Above a Charged Disk
	$$\vec{E} = \iint_S \frac{dQ'(\vec{R}-\vec{R}')}{4\pi\varepsilon_0|\vec{R}-\vec{R}'|^3}$$
	By definition,
	$$dQ' = \rho_Sds'$$
	and we can use cylindrical coordinates, so
	$$ds' = rdrd\phi$$
	The position vectors are given by
	\begin{align*}
		\vec{R} &= h\hat{a}_z \\
		\vec{R}' &= r\hat{a}_r
	\end{align*}
	Note that $r$ is constant. Then
	$$\vec{R}-\vec{R}' = -r\hat{a}_r + h\hat{a}_z = -r\cos\phi\hat{a}_x -r\sin\phi\hat{a}y + h\hat{a}_z$$
	And
	$$d\vec{E} = \frac{\rho_Srdrd\phi}{4\pi\varepsilon_0(r^2+h^2)^{3/2}}$$
	By symmetry, $\vec{E}$ must be a multiple of $\hat{a}_z$. Integrating gives
	$$\frac{\rho_s}{2\varepsilon_0}\left(\frac{h}{|h|}-\frac{h}{\sqrt{a^2+h^2}}\right)$$
	Then as $h\rightarrow0$ or $a\rightarrow\infty$, this simplifies to $\frac{\rho_S}{2\varepsilon_0}$. Then a capacitor can be build by placing two sheets of oppositely charged plates next to each other in parallel.
\end{ex}
	

\end{document}
