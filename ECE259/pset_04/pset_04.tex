\documentclass[answers]{exam}
\usepackage{../../template}
\author{niceguy}
\title{Problem Set 4}
\begin{document}
\maketitle

\begin{questions}

\question{Consider a region with a uniform electrostatic field of intensity $E$. If the electric scalar potential at the point $A$ is zero, the potential at the point $B$ equals}

\begin{solution}
	$$V_B = -Ed\cos\alpha$$
\end{solution}

\question{A point charge $Q$ is situated in free space. The line integral of the electric field intensity vector $\vec{E}$ due to this charge along the contour $C$, composed of two circular parts of radii $a$ and $2a$, respectively, and two radial parts of length $a$, amounts to}

\begin{solution}
	$\vec{E}$ is conservative, so the integral is 0.
\end{solution}

\question{What happens to electric potentials and voltages in an electrostataic system after a new reference point is adopted for the potential?}

\begin{solution}
	Potentials change by the same value and voltages remain unchanged.
\end{solution}

\question{The electrostatic potential $V$ in a aregion is a function of the rectangular coordinate $x$ only. Consider the electric field intensities at points $A,B,C,D$, and $E$. The largest field intensity is at point}

\begin{solution}
	It is where the slope is maximum, which is $C$.
\end{solution}

\question{Consider an electrostatic field in a region of space and the following two statements. Which of the statements is true?}

\begin{parts}
	\part{If the electric scalar potential at a point in the region is zero, then the electric field vector at that point must be zero as well.}
	\part{If the electric field vector at a point in zero, then the potential at the same point must be zero.}
\end{parts}

\begin{solution}
	None of the statements are true. The statements simply discuss if there is any implication between $x=0$ and $x'=0$, where obvious counterexamples can be found.
\end{solution}

\question{An uncharged thin metallic rod is introduced into a uniform electrostatic field, of intensity vector $\vec{E}_0$, in free space, such that it is either perpendicular or parallel to $\vec{E}_0$. The rod affects the original field}

\begin{solution}
	Less in case (a). As the rod is a conductor, the electric field becomes close to zero, depending on conductivity.
\end{solution}

\question{A uniform electric field, of intensity vector $\vec{E}_0$, is established in the air-filled space between two metallic electrodes. If an uncharged (thick) metallic slab is then inserted in this space, without touching the electrodes, the electric field intensity vector in region 3 in the new electrostatic state is}

\begin{solution}
	$$E = \frac{V}{d}$$
	where $d$ is the distance. Substituting $V$ with $\frac{V}{2}$ and $d$ with $\frac{d}{3}$ gives
	$$\vec{E}_3 = \frac{3\vec{E}_0}{2}$$
\end{solution}

\question{A negatively charged small body is situated inside an uncharged spherical metallic shell. The distribution of induced charges on the outer surface of the shell can be represented as in}

\begin{solution}
	A
\end{solution}

\question{In order to protect body $B$ from the electrostatic field due to a charged body $A$, an ungrounded closed metallic screen is introduced. The protection is achieved for}

\begin{solution}
	b only.
\end{solution}

\question{point charge $Q$ is situated in air at a height $h$ above a grounded conducting plane. Relative to the plane, the electric force on this charge is}

\begin{solution}
	Always attractive.
\end{solution}

\question{The electrostatic potential $V$ in a region is a function of a single rectangular coordinate $x$, $V(x)$ and is shown in the figure below. Sketch the components of the electric field intensity $E$ in this region}

\begin{solution}
	\begin{tabular}{c|c|c|c|c|c|c|c|c|c|c}
		\hline
		$x$ & 0-1 & 1-2 & 2-3 & 3-4 & 4-5 & 5-6 & 6-7 & 7-8 & 8-9 & 9-10 \\
		\hline
		$E(x)$ & 2 & 1 & 0 & -1 & -2 & -2 & -1 & 0 & 1 & 2 \\
		\hline
	\end{tabular}
\end{solution}

\question{For the three charges in the figure below, calculate the electric potential at points defined by}

\begin{parts}
	\part{(0,0,2)}
	\part{(1,1,1)}
\end{parts}

\begin{solution}
	$(0,0,2)$:
	$$\frac{2\times10^{-6}k}{1} - \frac{2\times10^{-6}k}{\sqrt{5}} + \frac{10^{-6}k}{\sqrt{5}} = 10^{-6}k\left(2-\frac{1}{\sqrt{5}}\right)$$
	$(1,1,1)$:
	$$\frac{(1-2+2)10^{-6}k}{\sqrt{2}} = \frac{10^{-6}k}{\sqrt{2}}$$
\end{solution}

\question{For the semi-circular line charge in the figure below, the electric field at an arbitrary point on the $z$-axis has an $x$ and a $z$ component (confirm). Find the 𝑧-component of the field from the potential $V(0,0,𝑧)$. Can you find the $x$-component too using $V (0,0,z)$?}

\begin{solution}
	By symmetry, there is no $y$ component.
	\begin{align*}
		V(0,0,z) &= \int \frac{kdQ}{r} \\
			 &= \int_{-\frac{\pi}{2}}^\frac{\pi}{2} \frac{ka\rho_ld\phi}{\sqrt{a^2+z^2}} \\
			 &= \frac{ka\rho_l\pi}{\sqrt{a^2+z^2}} \\
			 &= \frac{kQ}{\sqrt{a^2+z^2}}
	\end{align*}
	Then the $z$-component is
	\begin{align*}
		E_z &= -\frac{dV(0,0,z)}{dz} \\
		    &= \frac{kQ}{2(a^2+z^2)^{3/2}} \times 2z \\
		    &= \frac{kQz}{(a^2+z^2)^{3/2}}
	\end{align*}
	The $x$-component cannot be found.
\end{solution}

\question{Two point charges $Q_1 = 7\unit{\mu.C}$, and $Q_2 = -3\unit{\mu.C}$, are located on two non-adjacent vertices of a square contour $a=15\unit{cm}$ on a side. Find the voltage between any of the remaining two vertices of the square and the square center.}

\begin{solution}
	By symmetry, both voltages are the same.
	$$\frac{(7-3)\times10^{-6}k}{0.15} - \frac{(7-3)\times10^{-6}k}{0.15\div\sqrt{2}} = -99.3\unit{kV}$$
\end{solution}


\end{questions}
\end{document}
