\documentclass[12pt]{article}
\usepackage{../../template}
\author{niceguy}
\title{Lecture 11}
\begin{document}
\maketitle

\section{Material Classification}
If an external electric field $\vec{E}$ is applied to a material then the excess or mobile charges will be pushed along by the field through the electric force
$$\vec{F}_E = q\vec{E} = -e\vec{E}$$
Which leads to a current. Based on this, we can classify materials as \emph{conductors}, \emph{semiconductors}, and \emph{dielectircs}.

\section{Point form of Ohm's Law and Conductivity}

$$I = \iint_S \vec{J}\dot d\vec{S}$$

Where $\vec{J}$ is the current density.

$$\vec{J} = \sigma\vec{E}$$

Which relates to Ohm's Law
$$I = GV = \frac{V}{R}$$

Rearranging,
$$\sigma = \frac{N_ee^2\tau}{m_e}$$
Where $N_e$ is electron density and $\tau$ is mean free time (in seconds). \\
Generally, conductivity $\sigma$ is inversely proportional to temperature $T$. Resistivity is just the inverse of conductivity,
$$\rho = \frac{1}{\sigma}$$


\end{document}
