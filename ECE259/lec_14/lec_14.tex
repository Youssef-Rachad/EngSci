\documentclass[12pt]{article}
\usepackage{../../template}
\author{niceguy}
\title{Lecture 14}
\begin{document}
\maketitle

\section{Dielectric Strength}

The \emph{dielectric strength} of a material is the maximum field it can withstand before a current flows. This is also called \emph{breakdown}. Consider the band gap: once sufficient energy is supplied, electrons can jump from the valance band to the conducting band. In other words, the field is strong enough to overcome the attractive force between the nucleus and its electrons. The electrons become detached, and a current flows.

\section{Boundary Conditions for the Electric Field}

Application: Optical Fibre. Its low conuctivity reduces conductive power loss.
\end{document}
