\documentclass[12pt]{article}
\usepackage{../../template}
\author{niceguy}
\title{Lecture 17}
\begin{document}
\maketitle

\section{Capacitors}

\begin{ex}
	Consider the capacitors half filled with a dielectric material (in both senses). It is know that the charge density on the plates is non-uniform. Explain why this is the case, and find an expression for the total capacitance of this structure. \\
	If the bottom half of the capacitor is filled with a dielectric material, then $\vec{D}$ is constant, and is equal to $\rho_s$. Then
	\begin{align*}
		\Delta V &= -\int \vec{E}\cdot d\vec{l} \\
			 &= -E_{\text{air}}\times\frac{d}{2} - E_{\text{diel}}\times\frac{d}{2} \\
			 &= -\frac{\rho_s}{\varepsilon_0}\frac{d}{2} - \frac{\rho_s}{\varepsilon_0\varepsilon_r}\frac{d}{2} \\
			 &= -Q\times\frac{d}{2\varepsilon_0s} - Q\times\frac{d}{2\varepsilon_0\varepsilon_rs} \\
			 &= -\frac{Q}{C_1} - \frac{Q}{C_2}
	\end{align*}
	Thus the capacitance of capacitors in series is given by
	$$C = \left(\frac{1}{C_1}+\frac{1}{C_2}\right)^{-1}$$
	If they are in parallel,note that tangential $\vec{E}$ is constant. Hence
	$$\rho_{\text{air}} = D_{\text{air}} = \varepsilon_0\frac{V}{d}$$
	$$\rho_{\text{diel}} = D_{\text{diel}} = \varepsilon_0\varepsilon_r\frac{V}{d}$$
	And total capacitance is
	$$C = C_1 + C_2$$
	where
	$$C_1 = \frac{\varepsilon_0s}{2d}$$
	and
	$$C_2 = \frac{\varepsilon_0\varepsilon_rs}{2d}$$
\end{ex}

\section{Electrostatic Potential Energy}

Energy can be stored in a charge distribution if there is positive work done in assembling it.
$$W_e = \frac{1}{2}\sum_{i=1}^NQ_iV_i$$
Note the factor of $1/2$
$$W_e = \frac{1}{2}\sum_{i=1}^NQ_iV_i$$
Note the factor of $\frac{1}{2}$. Consider arbitrary $i$ and $j$. Through the sum, we are double counting the effects of $Q_i$ on $Q_j$ and vice versa. However, one charge (WLOG $Q_i$) is introduced first, so it does not "feel" the effect of $Q_j$. We divide by 2 to correct for double counting. \\
If charge distribution is continuous,
$$W = \int_\Omega \rho Vd\Omega$$

\begin{ex}
	What is the energy stored within a parallel plate capacitor? \\
	\begin{align*}
		W &= \frac{1}{2} \iiint \vec{D}\cdot\vec{E}dV \\
		  &= \frac{1}{2} \iiint \varepsilon_0\varepsilon_r |\vec{E}|^2dv \\
		  &= \frac{1}{2}\frac{\rho_s^2}{\varepsilon_0\varepsilon_r}sd \\
		  &= \frac{1}{2} \frac{Q^2d}{\varepsilon_0\varepsilon_rs} \\
		  &= \frac{1}{2} \frac{Q^2}{C}
	\end{align*}
	Thus we also have the relation
	$$C = \frac{1}{2} \frac{Q^2}{W} = \frac{2W}{V^2}$$
	The other approach is
	\begin{align*}
		W &= \frac{1}{2} \iint\rho_sVds \\
		  &= \frac{1}{2} \rho_sVs \\
		  &= \frac{1}{2} QV
	\end{align*}
\end{ex}

\end{document}
