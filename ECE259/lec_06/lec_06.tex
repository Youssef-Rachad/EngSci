\documentclass[12pt]{article}
\usepackage{../../template}
\author{niceguy}
\title{Lecture 6}
\begin{document}
\maketitle

\section{Review}

We can calculate electric field due to static charge for discrete charges

$$\vec{E}_{\text{total}} = \sum_i \frac{Q_i}{4\pi\varepsilon_0|\vec{R}-\vec{R}'|^3}(\vec{R}-\vec{R}')$$

and continuous charges

$$\vec{E}_{text{total}} = \int \frac{dQ'}{4\pi\varepsilon_0|\vec{R}-\vec{R}'|^3}(\vec{R}-\vec{R}')$$

\section{Fundamental Postulates of Electrostatics}

In differential form, they are

$$\vec{\nabla}\times\vec{E} = 0$$
$$\vec{\nabla}\cdot\vec{D} = \rho_v$$

or in integral form

$$\oint_C \vec{E}\cdot d\vec{l} = 0$$
$$\oiint_S \vec{D}\cdot d\vec{S} = Q_{\mathrm{enc}}$$

where the first equation comes from the fact that

$$\vec{E} = \vec{\nabla}V$$

and the curl of a gradient is zero. \\

$\varepsilon$ is the electrical permittivity of the material, with

$$\varepsilon = \varepsilon_r\varepsilon_0$$

The value of $\varepsilon_r$ in vacuum is 1, and it is 1.0006 in air, which is often approximated as 1. This relates $\vec{D}$ and $\vec{E}$

$$\vec{D} = \varepsilon\vec{E}$$

It is possible for $\varepsilon$ to vary over a region, e.g. for an anisotropic crystal.

\begin{ex}
	For the electric flux density given by $\vec{D} = \frac{5}{4}\left(R^2-\frac{1}{R^2}\right)\hat{a}_R$ for $R<2$m. Use Gauss's law to determine the volume charge density in this region. \\
	The gradient in spherical coordinates is given by
	$$\vec{\nabla}\cdot\vec{A} = \frac{1}{R^2}\frac{\partial(R^2A_R)}{\partial R} + \frac{1}{R\sin\theta}\frac{\partial(\sin\theta A_\theta)}{\partial \theta} + \frac{1}{R\sin\theta}\frac{\partial A_\phi}{\partial \phi}$$
	Only the first term is nonzero, as $\vec{D}$ is a function of $R$ only. Substituting, we have
	\begin{align*}
		\rho_v &= \frac{1}{R^2}\frac{\partial}{\partial R} \left(\frac{5}{4}(R^4-1)\right) \\
		       &= \frac{1}{R^2} (5R^3) \\
		       &= 5R
	\end{align*}
\end{ex}

Note that from Coulomb's law, we have

$$\vec{D} = \varepsilon\vec{E} = \frac{Q}{4\pi R^2}\hat{a}_R$$

Then integrating over the surface of a sphere,

$$\oiint_S \vec{D}\cdot d\vec{S} = \oiint_S \frac{Q}{4\pi R^2} dS = Q$$
\end{document}
