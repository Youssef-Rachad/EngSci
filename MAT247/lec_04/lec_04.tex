\documentclass[12pt]{article}
\usepackage{../../template}
\author{niceguy}
\title{Lecture 4}
\begin{document}
\maketitle

\section{Inverse of a Matrix}

Denote columns of $A^{-1}$ by $w_i$. Then observing
$$v_i = Ae_i$$
we can see
$$w_j = A^{-1}e_j \Leftrightarrow Aw_j = e_j$$
That is, $w_j$ solves the equation $Ax=e_j$. Thus by Cramer's rule,
$$A^{-1}_{ij} = \frac{\det(v_1,\dots,v_{i-1},e_j,v_{i+1},\dots,v_n)}{\det(A)}$$
Using cofactor expansion along the $j$th column
$$A^{-1}_{ij} = \frac{(-1)^{i+j}\det(A^{[ji]})}{\det(A)}$$
Note that $i$ and $j$ are flipped for the cofactor, as it is the $j$th row that is deleted.

\section{More Properties}

For $T\in\mathcal{L}(V)$ we defined
$$\det(T) = \det(A)$$
giving the properties
$$\det(TS) = \det(T)\det(S)$$
$$\det(T^{-1}) = \det(T)^{-1}$$
$$\det(T') = \det(T)$$

If $W\subseteq V$ is $T$-invariant then
$$\det(T) = \det(T|_W)\det(\bar{T})$$
where $\bar{T} \in \mathcal{L}(V/W)$ is $\bar{T}(v+W) = Tv+W$

\section{Characteristic Polynomials}

Let $T\in\mathcal{L}(V)$

\begin{thm}
	A scalar $\lambda\in\F$ is an eigenvalue of $T$ iff $\det(\lambda I-T) = 0$
\end{thm}

Proof: \\
The definition of an eigenvalue $\lambda$ is that $Tv=\lambda v$. Rearranging this, $v$ is in the nullspace of $(\lambda I-T)$, which is equivalent to it being noninvertible, giving it a determinant of 0. \\
To rephrase this, $\lambda$ is an eigenvalue iff $\lambda$ is a root of
$$q(z) = \det(zI - T)$$

\begin{defn}
	For $T\in\mathcal{L}(V)$, the polynomial
	$$q(z) = \det(zI-T)$$
	is called the characteristic polynomial. Similarly, for $A\in M_{n\times n}(\F)$ we call
	$$q(z) = \det(zI-A)$$
	the characteristic polynomial.
\end{defn}

Remark: \\
The definition works for any field $\F$
$$p(z) = \sum_{i=0}^n a_iz^i$$
where $a_i\in\F$. Note that the degree of the polynomial is at most $n$ (based on the permutations).
For $\F = \C$, we can use the fundamental theorem of algebra to factorise $q(z)$
$$q(z) = \prod_i (z-\lambda_i)$$
where some eigenvalues may repeat. Then this is equivalent to the set of eigenvalues by the theorem above. \\
Then for any square matrix $A$, for any upper triangular matrix with $A'_{ii} = \lambda_i$, we see
$$\det(zI-A) = \det(zI-A')$$
Remark: \\
If $A$ has block upper triangular form
$$A = \begin{pmatrix} A' & * \\ 0 & A''\end{pmatrix}$$
then
$$q_A(z) = q_{A'}(z)q_{A''}(z)$$
since
$$zI-A = \begin{pmatrix} zI-A' & * \\ 0 & zI-A''\end{pmatrix}$$

\section{Cyclic Vectors}

Suppose $T\in\mathcal{L}(V),v\neq0\in V, v_k = T^{k-1}v$. There is a smallest $k$ such that $v_1,\dots,v_{k+1}$ are linearly dependent, The subspace $W$ spanned by $v_1,\dots,v_k$ is $T$-invariant, as $v_i$ maps to $v_{i+1}$ except for $i=k$, where it maps to a linear combination of $v_i,\dots,v_k$. \\
We call $v\in V$ a cyclic vector for $T$ if $v_1,\dots$ spans all of $V$. In this case, $v_1,\dots,v_k$ is a basis of $V$. The matrix of $T$ in this basis is
$$A_{\cdot,i} = \begin{cases} e_{i+1} & i\neq k \\ b & i = k\end{cases}$$
where $A_{\cdot,i}$ is the $i$th column of $A$, and
$$b = \sum_{i=1}^k -a_ie_i$$
where
$$Tv_k = \sum_{i=1}^k -a_iv_i$$
\end{document}
