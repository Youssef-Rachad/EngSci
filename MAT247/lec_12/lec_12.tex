\documentclass[12pt]{article}
\usepackage{../../template}
\author{niceguy}
\title{Lecture 12}
\begin{document}
\maketitle

\section{Adjoints}

Let $\F = \R$ or $\F = \C$, and $T \in \mathcal L(V)$. Using induction, one can prove that
$$\langle T^n(v),w \rangle = \langle v,(T^*)^n(w) \rangle \forall n \in \N$$
Extending this to the minimal polynomial,
\begin{align*}
	0 &= \langle p(T)v,w \rangle \\
	  &= \langle v,\overline{p}(T^*)w \rangle
\end{align*}
However, this is true for arbitrary $v,w \in V$. Hence $\overline{p}$ is the minimal polynomial for $T^*$. (Obviously $\overline{p}(T^*) = 0$. If there is a polynomial with a smaller degree, one obtains a $p'$ of that lower degree such that $p'(T) = 0$, which is a contradiction, as $p$ is the minimal polynomial). \\
Recall with a orthonormal basis, if $A$ is the matrix of $T$, then the matrix of $T^*$ is $\overline{A}^t$. Assuming this, consider the characteristic polynomial
$$q_T(z) = \det(zI-T) = \det(zI-A)$$
$$q_{T^*}(z) = \det(zI-T^*) = \det(zI-\overline{A}^t) = \det\left((zI-\overline{A})^t\right) = \det(zI-\overline{A}) = \overline{\det(\overline{z}I-A)}$$
Then
$$q_{T^*}(z) = \overline{q_T(\overline{z})}$$
Now, let $J$ be the Jordan Normal Form of $T$. Then
\begin{align*}
	A &= CJC^{-1} \\
	\overline{A} &= \overline{C}\overline{J}\overline{C^{-1}} \\
	\overline{A}^t &= \overline{C^{-1}}^t\overline{J}^t\overline{C}^t \\
		       &= \overline{C^{-1}}^t\overline{S}\overline{J}\overline{S^{-1}}\overline{C}^t
\end{align*}
Undoing the change of basis, the Jordan Normal Form of $T^*$ is $\overline{J}$. \\
Suppose $T^* = T$. Then
\begin{enumerate}
	\item if $W \subseteq V$ is $T$ invariant, then $W^\perp$ is alto $T$ invariant
	\item The eigenvalues of $T$ are real
	\item T has at least one eigenvector
\end{enumerate}

\begin{thm}
	$T$ is diagnoalisable if $T=T^*$.
\end{thm}

\begin{proof}
	By 3 there is an eigenvector $v_1$ of $T$. Then $W = fv_1 \forall f \in \F$ is a $T$ invariant subspace. So $W^\perp$ is also $T$ invariant, and
	$$T|_{W^\perp} \in \mathcal L(W^\perp)$$
	Then
	$$\left(T|_{W^\perp}\right)^* = \left(T|_{W^\perp}\right)$$
	By induction, we get $v_1,v_2,\dots,v_n$ where $n$ is the dimension of $V$. All $v_i$ are eigenvectors, so $T$ is diagonalisable.
\end{proof}
\end{document}
