\documentclass[answers]{exam}
\usepackage{../../template}
\author{niceguy}
\title{Homework 2}
\begin{document}
\maketitle

\begin{questions}

\question{If a letter is chosen at random from the English alphabet, find the probability that the letter}

\begin{parts}
\part{is a vowel exclusive of $y$;}
\part{is listed somewhere ahead of the letter $j$;}
\part{is listed somewhere after the letter $g$}
\end{parts}

\begin{solution}
	$\frac{5}{26}, \frac{9}{26}, \frac{19}{26}$
\end{solution}

\question{Prove that
	$$P(A'\cap B') = 1 + P(A\cup B) - P(A) - P(B)$$
}

\begin{solution}
	\begin{align*}
		P(A'\cap B') &= P((A\cup B)') \\
			     &= 1 - P(A\cup B) \\
			     &= 1 + P(A\cap B) - P(A) - P(B)
	\end{align*}
	where the first equality comes from De Morgan.
\end{solution}

\question{Pollution of the reivers in the United States has been a problem for many years. Consider the following events:
	$$A: \text{ the river is polluted}$$
	$$B: \text{ a sample of water tested detects pollution}$$
	$$C: \text{ fishing is permitted}$$
Assume $P(A)=0.3,P(B|A)=0.75,P(B|A')=0.20,P(C|A\cap B)=0.20,P(C|A'\cap B)=0.15,P(C|A\cap B')=0.80$, and $P(C|A'\cap B')=0.90$.}

\begin{parts}
\part{Find $P(A\cap B\cap C)$}
\part{Find $P(B'\cap C)$}
\part{Find $P(C)$}
\part{Find the probability that the river is polluted, given that fishing is permitted and the sample tested did not detect pollution.}
\end{parts}

\begin{solution}
	\begin{align*}
		P(A\cap B\cap C) &= \frac{P(A\cap B\cap C)}{P(A\cap B)}\times P(A\cap B) \\
				 &= P(C|A\cap B)P(B|A)P(A) \\
				 &= 0.20\times0.75\times0.3 \\
				 &= 0.045
	\end{align*}
	Given $P(A) = 0.3$ and $P(B|A) = 0.75$, we can deduce $P(A\cap B) = 0.225$. Similarly, from $P(B|A')$ we obtain $P(A'\cap B) = 0.14$. Summing we have $P(B) = 0.365$. This gives $P(A\cap B') = 0.075$ and $P(A'\cap B') = 0.56$. From $P(C|A\cap B)$ we get $P(A\cap B\cap C) = 0.045$. Similarly, $P(A'\cap B\cap C) = 0.021, P(A\cap B'\cap C) = 0.06, P(A'\cap B'\cap C) = 0.504$. Summing we have $P(C) = 0.63$ and $P(B'\cap C) = 0.564$. Then finally
	\begin{align*}
		P(A|C\cap B') &= \frac{P(A\cap B'\cap C)}{P(B'\cap C)} \\
			      &= \frac{0.06}{0.564} \\
			      &= \frac{5}{47} \\
			      &= 0.106
	\end{align*}
\end{solution}

\question{Suppose the diagram of an electrical system is as given in Figure 2.10. What is the probability that the system works? Assume the components fail independently.}

\begin{solution}
	$$0.95\times(1-0.2\times0.3)\times0.9 = 0.8037$$
\end{solution}

\question{A paint-store chain produces and sells latex and semigloss paint. Based on long-range sales, the probability that a customer will purchase latex paint is 0.75. Of those that purchase latex paint, 60\% also purchase rollers. But only 30\% of semigloss paint buyers purchase rollers. A randomly selected buyer purchases a roller and a can of paint. What is the probability that the paint is latex?}

\begin{solution}
	Let $A$ be the purchase of latex paint, $B$ be the purchase of rollers, and $C$ the the purchase of semigloss paint. Then
	\begin{align*}
		P(A|B) &= \frac{P(A\cap B)}{P(B)} \\
		       &= \frac{P(B|A)P(A)}{P(B)} \\
		       &= \frac{0.6\times0.75}{P(B|A)P(A) + P(B|C)P(C)} \\
		       &= \frac{0.45}{0.45 + 0.3\times0.25} \\
		       &= \frac{0.45}{0.525} \\
		       &= 0.857
	\end{align*}
\end{solution}

\question{Let $W$ be a random variable giving the number of heads minus the number of tails in three tosses of a coin. List the elements of the sample space $S$ for the three tosses of the coin and to each sample point assign a value $w$ of $W$.}

\begin{solution}
	$$S = \{HHH(3), HHT(1), HTH(1), HTT(-1), THH(1), THT(-1), TTH(-1), TTT(-3)\}$$
\end{solution}

\question{The total number of hours, measured in units of 100 hours, that a family runs a vacuum cleaner over a period of one year is a continuous random variable $X$ that has the density function
	$$f(x) = \begin{cases} x & 0<x<1 \\ 2-x & 1\leq x<2 \\ 0 & \text{elsewhere}\end{cases}$$
Find the probability that over a period of one year, a family runs their vacuum cleaner}

\begin{parts}
\part{less than 120 hours}
\part{between 50 and 100 hours}
\end{parts}

\begin{solution}
	For less than 120 hours, $x \in [0,1.2]$, so the probability is
	$$0.5 + (1+0.8)\times0.1 = 0.68$$
	For between 50 and 100 hours, the probability is
	$$0.5\times0.75 = 0.375$$
\end{solution}

\question{Suppose it is known from large amounts of historical data that $X$, the number of cars that arrive at a specific intersection during a 20-second time period, is characterized by the following discrete probability function:
	$$f(x) = e^{-6}\frac{6^x}{x!} \forall x \in \N$$
}

\begin{parts}
\part{Find the probability that in a specific 20-second time period, more than 8 cars arrive at the intersection.}
\part{Find the probability that only 2 cars arrive.}
\end{parts}

\begin{solution}
	For more than 8 cars to arrive, the probability is
	$$1-\sum_{i=0}^8f(i) = 0.153$$
	For two cars to arrive, the probability is
	$$f(2) = 0.0446$$
\end{solution}

\question{Let $X$ denote the reaction time, in seconds, to a certain stimulus and $Y$ denote the temperature ($^\circ$F) at which a certain reaction starts to take place. Suppose that two random variables $X$ and $Y$ have the joint density
	$$f(x,y) = \begin{cases} 4xy & 0<x<1,0<y<1 \\ 0 & \text{elsewhere}\end{cases}$$
Find}

\begin{parts}
\part{$P(0\leq X\leq\frac{1}{2}\cap\frac{1}{4}\leq Y\leq\frac{1}{2})$}
\part{$P(X<Y)$}
\end{parts}

\begin{solution}
	The first probability is given by
	$$\int_0^{\frac{1}{2}}\int_{\frac{1}{4}}^{\frac{1}{2}} 4xy dydx = 0.046875$$
	For the second probability, note that $P(X=Y)=0$, so by symmetry
	$$P(X<Y) = 0.5$$
\end{solution}

\question{The joint probability density function of the random variables $X$, $Y$, and $Z$ is
	$$f(x,y,z) = \begin{cases} \frac{4xyz^2}{9} & 0<x,y<1,0<z<3 \\ 0 & \text{elsewhere}\end{cases}$$
Find}

\begin{parts}
\part{the joint marginal density function of $Y$ and $Z$}
\part{the marginal density of $Y$}
\part{$P(\frac{1}{4}<X<\frac{1}{2},Y>\frac{1}{3},1<Z<2)$}
\part{$P(0<X<\frac{1}{2}|Y=\frac{1}{4},Z=2)$}
\end{parts}

\begin{solution}
	The joint marginal density is
	$$g(y,z) = \int_0^1 \frac{4xyz^2}{9}dx = \frac{2yz^2}{9}$$
	The marginal density is
	$$h(y) = \int_0^3\int_0^1 \frac{4xyz^2}{9}dxdz = 2y$$
	The first desired probability is
	$$\int_1^2\int_{\frac{1}{3}}^1\int_{\frac{1}{4}}^{\frac{1}{2}} \frac{4xyz^2}{9}dxdydz = 0.0432$$
	The second desired probability is
	$$\frac{\int_0^\frac{1}{2} \frac{4x}{9}dx}{g\left(\frac{1}{4},2\right)} = \frac{1}{18}\div\frac{2}{9} = \frac{1}{4}$$
\end{solution}

\end{questions}
\end{document}
