\documentclass[12pt]{article}
\usepackage{../../template}
\author{niceguy}
\title{Lecture 24}
\begin{document}
\maketitle

\section{Confidence Intervals}

Recall with $n$ IID samples, an observec mean $\overline{X}$, a known variance $\sigma^2$ and statistic
$$Z = \frac{\overline{X}-\mu}{\frac{\sigma}{\sqrt{n}}}$$
By the central limit theorem, $Z$ has $n(z;0,1)$. We set
$$z_\beta = -\Phi^{-1}(\beta)$$

\begin{align*}
	1 - \alpha &= P(-z_{\frac{\alpha}{2}} \leq Z \leq z_{\frac{\alpha}{2}}) \\
		   &= P\left(\overline{X} - \frac{z_{\frac{\alpha}{2}}\sigma}{\sqrt{n}} \leq \mu \leq \overline{X} + \frac{z_{\frac{\alpha}{2}}\sigma}{\sqrt{n}}\right) \\
		   &= P(\overline{X}_L \leq Z \leq \overline{X}_U)
\end{align*}

\subsection{Realised Confidence Interval}

The real confidence interval is then
$$\left[\overline{x} - \frac{z_{\frac{\alpha}{2}}\sigma}{\sqrt{n}}, \overline{x} + \frac{z_{\frac{\alpha}{2}}\sigma}{\sqrt{n}}\right]$$

\subsection{One Sided Confidence Interval}

There is also a one sided confidence interval
$$1 - \alpha = P(Z \leq z_\alpha)$$

where similarly, we set
$$z_\alpha = -\Phi^{-1}(\alpha)$$

and we have
$$1 - \alpha = P\left(\mu \leq \overline{X} + \frac{z_\alpha\sigma}{\sqrt{n}}\right)$$

\section{Confidence Interval with Unknown Variance}

If variance is not known, with normal samples, then setting
$$T = \frac{\overline{X}-\mu}{\frac{s}{\sqrt{n}}}$$
where
$$S^2 = \frac{1}{n-1} \sum_i\left(X_i-\overline{X}\right)^2$$
Then $T$ has a t distribution. For $\beta < 0.5$, letting $H(t)$ be the cumulative distirbution function, we can define $t_\beta$ such that
$$t_\beta = H^{-1}(\beta)$$
Then similarly,

\begin{align*}
	1 - \alpha &= P\left(-t_{\frac{\alpha}{2}} \leq T \leq t_{\frac{\alpha}{2}}\right) \\
		   &= P\left(-t_{\frac{\alpha}{2}} \leq \frac{\overline{X}-\mu}{\frac{S}{\sqrt{n}}} \leq t_{\frac{\alpha}{2}}\right) \\
		   &= P\left(\overline{X} - \frac{t_{\frac{\alpha}{2}}S}{\sqrt{n}} \leq \mu \leq \overline{X} + \frac{t_{\frac{\alpha}{2}}S}{\sqrt{n}}\right)
\end{align*}

Then given $\overline{x}$, its realisation is
$$\left[\overline{x} - \frac{t_{\frac{\alpha}{2}}S}{\sqrt{n}}, \overline{x} + \frac{t_{\frac{\alpha}{2}}S}{\sqrt{n}}\right]$$

\begin{ex}
	Let $n=7$, with normal samples. The observed mean $\overline{x}$ is -3, and the observed variance $s^2 = 2$. Setting $\alpha=0.1$ for a 90\% confidence interval,
	$$t_{0.05} = -1.9$$
	so the realised confidence interval is
	$$\left[-3-1.9\times\frac{2}{\sqrt{7}},-3+1.9\times\frac{2}{\sqrt{7}}\right] = [-3.52,-2.48]$$
\end{ex}

\section{Standard Error}

For samples $X_1,\dots,X_n$, and
$$Z = \frac{\overline{X}-\mu}{\frac{\sigma}{\sqrt{n}}}$$
with normal distribution, $\overline{X}$ has a standard error of $\frac{\sigma}{\sqrt{n}}$. The width of the confidence interval is proportional to this.

\section{Prediction Intervals}

With samples $X_1,\dots,X_n$ that are normal, let $X_0$ be a new observation. Now $\overline{X}$ is a good point estimation of $X_0$. The error is $X_0 - \overline{X}$, and the variance of the error is $\sigma^2 + \frac{\sigma^2}{n}$. For the statistic
$$Z = \frac{X_0-\overline{X}}{\sigma\sqrt{1+\frac{1}{n}}}$$
$Z$ has a normal distribution, then

\begin{align*}
	1 - \alpha &= P(-z_{\frac{\alpha}{2}} \leq Z \leq z_{\frac{\alpha}{2}}) \\
		   &= P\left(\overline{X} - z_{\frac{\alpha}{2}}\sigma\sqrt{1+\frac{1}{n}} \leq X_0 \leq \overline{X} + z_{\frac{\alpha}{2}}\sigma\sqrt{1+\frac{1}{n}}\right)
\end{align*}

Where
$$z_{\frac{\alpha}{2}} = -\Phi^{-1}\left(\frac{\alpha}{2}\right)$$
\end{document}
