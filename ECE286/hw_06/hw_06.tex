\documentclass[answers]{exam}
\usepackage{../../template}
\author{niceguy}
\title{Homework 6}
\begin{document}
\maketitle

\begin{questions}

\question{Let $X$ be a binomial random variable with probability distribution
	$$f(x) = \begin{cases} \binom{3}{x}\left(\frac{2}{5}\right)^x \left(\frac{3}{5}\right)^{3-x} & x=0,1,2,3 \\ 0 & \text{elsewhere} \end{cases}$$
	Find the probability distribution of the random variable $Y=X^2$}

\begin{solution}
	$$g(y) = \begin{cases} \binom{3}{\sqrt{y}} \left(\frac{2}{5}\right)^{\sqrt{y}} \left(\frac{3}{5}\right)^{3-\sqrt{y}} & y=0,1,4,9 \\ 0 & \text{elsewhere} \end{cases}$$
\end{solution}

\question{A dealer’s profit, in units of \$5000, on a new automobile is given by $Y = X^2$ , where X is a random variable having the density function
$$f(x) = \begin{cases} 2(1-x) & 0<x<1 \\ 0 & \text{elsewhere} \end{cases}$$}

\begin{parts}
	\part{Find the probability density function of the random variable $Y$.}
	\part{Using the density function of Y , find the probability that the profit on the next new automobile sold by this dealership will be less than \$500.}
\end{parts}

\begin{solution}
	\begin{align*}
		G(y) &= P(Y\leq y) \\
		     &= P(\sqrt{y} \leq X \leq \sqrt{y})
	\end{align*}
	For $0 < y < 1$, this gives
	$$G(y) = \int_0^{\sqrt{y}} 2(1-x)dx = 2\sqrt{y} - y$$
	And so
	$$g(y) = y^{-\frac{1}{2}} - 1$$
	in this range. The probability density function is then
	$$g(y) = \begin{cases} y^{-\frac{1}{2}} - 1 & 0 < y < 1 \\ 0 & \text{elsewhere} \end{cases}$$
	The desired probability is then $G(0.1) = 0.532$.
\end{solution}

\question{The random variables $X$ and $Y$, representing the weights of creams and toffees, respectively, in 1-kilogram boxes of chocolates containing a mixture of creams, toffees, and cordials, have the joint density function
$$f(x,y) = \begin{cases} 24xy & 0 \leq x \leq 1, 0 \leq y \leq 1, x+y \leq 1 \\ 0 & \text{elsewhere} \end{cases}$$
}

\begin{parts}
	\part{Find the probability density function of the random variable $Z = X + Y$.}
	\part{Using the density function of $Z$, find the probability that, in a given box, the sum of the weights of creams and toffees accounts for at least 1/2 but less than 3/4 of the total weight.}
\end{parts}

\begin{solution}
	Let $z' = \text{min}\{z,1\}$.
	\begin{align*}
		G(z) &= P(Z \leq z) \\
		     &= P(X + Y \leq z') \\
		     &= \int_0^{z'} \int_0^{z'-x} 24xy dydx \\
		     &= \int_0^{z'} 12x(z'-x)^2 dx \\
		     &= \int_0^{z'} (12x^3 - 24x^2 z' + 12xz'^2) dx \\
		     &= 3z'^4 - 8z'^4 + 6z'^4 \\
		     &= z'^4
	\end{align*}
	Differentiating,
	$$g(z) = \begin{cases} 4z^3 & 0 \leq z \leq 1 \\ 0 & \text{elsewhere} \end{cases}$$
	The required probability is then
	$$\left(\frac{3}{4}\right)^4 - \left(\frac{1}{2}\right)^4 = 0.254$$
\end{solution}

\question{Let $X_1$ and $X_2$ be independent random variables
each having the probability distribution
$$f(x) = \begin{cases} e^{-x} & x>0 \\ 0 & \text{elsewhere} \end{cases}$$
Show that the random variables $Y_1$ and $Y_2$ are independent when $Y_1=X_1+X_2$ and $Y_2 = \frac{X_1}{(X_1+X_2)}$.} \label{this}

\begin{solution}
	$$J  = \det\begin{pmatrix} \frac{dx_1}{dy_1} & \frac{dx_1}{dy_2} \\ \frac{dx_2}{dy_1} & \frac{dx_2}{dy_2} \end{pmatrix} = \det\begin{pmatrix} \frac{dy_1}{dx_1} & \frac{dy_1}{dx_2} \\ \frac{dy_2}{dx_1} & \frac{dy_2}{dx_2} \end{pmatrix}^{-1} = \det\begin{pmatrix} 1 & 1 \\ \frac{x_2}{(x_1+x_2)^2} & -\frac{x_1}{(x_1+x_2)^2} \end{pmatrix}^{-1} = -(x_1+x_2) = -y_1$$
	Then
	$$g(y_1,y_2) = f(y_1y_2, y_1(1-y_2))|J| = f(y_1y_2)f(y_1(1-y_2))y_1 = e^{-y_1y_2}e^{y_1(y_2-1)}y_1 = y_1e^{-y_1}$$
	Integrating,
	$$g(y_1) = \int_0^\infty g(y_1,y_2)dy_2 = \int_0^1 y_1e^{-y_1}dy_2 = y_1e^{-y_1}$$
	and
	$$g(y_2) = \int_0^\infty g(y_1,y_2)dy_1 = \int_0^\infty y_1e^{-y_1}dy_1 = 1$$
\end{solution}

\question{Show that the $r$th moment about the origin of the gamma distribution is
	$$\mu'_r = \frac{\beta^r\Gamma(\alpha+r)}{\Gamma(\alpha)}$$
}

\begin{solution}
	\begin{align*}
		\mu'_r &= E(X^r) \\
		       &= \int_{-\infty}^\infty x^rf(x)dx \\
		       &= \int_0^\infty \frac{x^{\alpha+r-1}e^{-\frac{x}{\beta}}}{\beta^\alpha\Gamma(\alpha)} dx \\
		       &= -\frac{x^{\alpha+r-1}e^{-\frac{x}{\beta}}}{\beta^{\alpha-1}\Gamma(\alpha)} \Big |_0^\infty + \int_0^\infty \frac{(\alpha+r-1)x^{\alpha+r-2}e^{-\frac{x}{\beta}}}{\beta^{\alpha-1}\Gamma(\alpha)} dx \\
		       &= \beta(\alpha+r-1)\mu'_{r-1}
	\end{align*}
	Then by induction, the statement obviously holds for $r=0$, as the left hand side gives the integral of the distribution, which is 1, and the right hand side obviously simplifies to 1. For the induction step, assume the statement holds for $r=k \in \N$. Then for $r=k+1$,
	$$\mu'_{k+1} = \beta(\alpha+k)\mu'_k = \beta(\alpha+k)\times\frac{\beta^k\Gamma(\alpha+k)}{\Gamma(\alpha)} = \frac{\beta^{k+1}\Gamma(\alpha+k+1)}{\Gamma(\alpha)}$$
	Thus the equation holds $\forall r \in \N$ by induction.
\end{solution}

\question{A random variable $X$ has the discrete uniform distribution
	$$f(x;k) = \begin{cases} \frac{1}{k} & $x=1,2,\dots,k$ \\ 0 & \text{elsewhere} \end{cases}$$
	Show that the moment-generating function of $X$ is
	$$M_X(t) = \frac{e^t(1-e^{kt})}{k(1-e^t)}$$
}

\begin{solution}
	\begin{align*}
		M_X(t) &= E(e^{tX}) \\
		       &= \sum_x e^{tx}f(x) \\
		       &= \sum_{x=1}^k \frac{e^{tx}}{k} \\
		       &= \frac{1}{k} \frac{e^{(k+1)t}-e^t}{e^t-1} \\
		       &= \frac{e^t(1-e^{kt})}{k(1-e^t)}
	\end{align*}
\end{solution}

\question{A random variable $X$ has the geometric distribution $g(x;p) = pq^{x-1}$ for $x=1,2,3,\dots$. Show that the moment-generating function of $X$ is
	$$M_X(t) = \frac{pe^t}{1-qe^t}, t < \ln q$$
and then use $M_X(t)$ to find the mean and variance of the geometric distribution.}

\begin{solution}
	\begin{align*}
		M_X(t) &= E(e^{tX}) \\
		       &= \sum_{x=1}^\infty e^{tx}pq^{x-1} \\
		       &= \frac{pe^t}{1-qe^t}
	\end{align*}
	Note that $M_X(t) = 1$, so substitution yields $p=1-q$.
	Then the mean is found by
	\begin{align*}
		\mu &= \mu'_1 \\
		    &= \frac{dM_X(t)}{dt} \Big |_{t=0} \\
		    &= \frac{p(1-qe^t)e^t+pqe^{2t}}{(1-qe^t)^2} \Big |_{t=0} \\
		    &= \frac{p(1-q)+pq}{(1-q)^2} \\
		    &= \frac{p}{(1-q)^2} \\
		    &= \frac{1}{p}
	\end{align*}
	The variance is found by
	\begin{align*}
		\sigma^2 &= \mu'_2 - \mu^2 \\
			 &= \frac{d}{dt} \frac{p(1-qe^t)e^t+pqe^{2t}}{(1-qe^t)^2} \Big |_{t=0} - \frac{1}{p^2} \\
			 &= \frac{(1-qe^t)^2p(e^t-2qe^{2t}+2qe^{2t})+2qe^t(1-qe^t)pe^t(1-qe^t+qe^t)}{(1-qe^t)^4} \Big |_{t=0} - \frac{1}{p^2} \\
			 &= \frac{(1-q)^2p(1-2q+2q)+2q(1-q)p(1-q+q)}{(1-q)^4} - \frac{1}{p^2} \\
			 &= \frac{p^3+2qp^2}{p^4} - \frac{1}{p^2} \\
			 &= \frac{p+2q}{p^2} - \frac{1}{p^2} \\
			 &= \frac{1-q+2q-1}{p^2} \\
			 &= \frac{q}{p^2}
	\end{align*}
\end{solution}

\question{A random variable $X$ has the Poisson distribution
	$$p(x;\mu) = e^{-\mu}\frac{\mu^x}{x!}, x \in \N$$
	Show that the moment-generating function of $X$ is
	$$M_X(t) = e^{\mu(e^t-1)}$$
Using $M_X(t)$, find the mean and variance of the Poisson distribution.}

\begin{solution}
	\begin{align*}
		M_X(t) &= E(e^{tX}) \\
		       &= \sum_{x=0}^\infty e^{tx-\mu}\frac{\mu^x}{x!} \\
		       &= e^{-\mu} \sum_{x=0}^\infty \frac{(\mu e^t)^x}{x!} \\
		       &= e^{-\mu} e^{\mu e^t} \\
		       &= e^{\mu(e^t-1)}
	\end{align*}
	The mean is then
	\begin{align*}
		\mu &= \mu'_1 \\
		    &= \frac{dM_X(t)}{dt} \Big |_{t=0} \\
		    &= \mu e^t\times e^{\mu(e^t-1)} \Big |_{t=0} \\
		    &= \mu
	\end{align*}
	And the variance is found by
	\begin{align*}
		\sigma^2 &= \mu'_2 - \mu^2 \\
			 &= \frac{d}{dt} \mu e^{\mu e^t + t - \mu} \Big |_{t=0} - \mu^2 \\
			 &= \mu(\mu e^t + 1) e^{\mu e^t + t - \mu} \Big |_{t=0} - \mu^2 \\
			 &= \mu(\mu + 1) - \mu^2 \\
			 &= \mu
	\end{align*}
\end{solution}

\question{The moment-generating function of a certain Poisson random variable $X$ is given by
	$$M_X(t) = e^{4(e^t-1)}$$
Find $P(\mu-2\sigma < X < \mu+2\sigma)$.}

\begin{solution}
	The mean and variance are both 4. Hence the distribution is
	$$\frac{e^{-4}4^x}{x!}$$
	and the desired probability is
	$$P(0 < X < 8)$$
	Then the probability is $0.931$.
\end{solution}

\question{If both $X$ and $Y$, distributed independently, follow exponential distributions with mean parameter 1, find the distributions of}

\begin{parts}
	\part{$U=X+Y$}
	\part{$V=\frac{X}{X+Y}$}
\end{parts}

\begin{solution}
	This is similar to question \ref{this}. Then the joint distribution is
	$$g(u,v) = ue^{-u}$$
	The distribution for $u$ is then
	$$g(u) = \int_0^1 ue^{-u}dv = ue^{-u}$$
	The distribution for $v$ is then
	$$g(v) = \int_0^\infty ue^{-u}du = 1$$
\end{solution}

\end{questions}
\end{document}
