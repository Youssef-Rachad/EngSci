\documentclass[12pt]{article}
\usepackage{../../template}
\author{Daniel Chua}
\title{Annotated Bibliography}
\begin{document}
\maketitle

\section{Classroom code-switching: three decades of research~\cite{classroom}}

\subsection{Brief Summary}

From the abstract, this paper reviews past research on classroom code-switching, raises difficulties and problems in such research, and concludes with future direction for research.

The author presented three purposes of code-switching

\begin{itemize}
    \item Ideational Functions
    \item Textual Functions
    \item Interpersonal Functions
\end{itemize}

In simple terms, ideational functions refer to using L1 languages to better explain concepts described in L2 terms. For example, a teacher may use a shared native language to "explain, elaborate or exemplify" scientific terms in English. Textual functions refer to topic shifts. For example, a math teacher in Hong Kong may use English to start the lesson, Cantonese to deal with late-comers, then switch back to English to continue the lesson. Finally, interpersonal functions "signal a shift in role-relationships" and "appeal to shared cultural values or institutional norms".

\subsection{Detailed Discussion}

The functions as described above give us the main purposes of code-switching in the classroom context, and how code-switching can be used to express different things, such as academic content, topic, or interpersonal relationships and cultural values. This signifies code-switching can be as simple as using a more familiar tongue to explain abstract concepts, or as complicated as building a sense of identity. An example given in the text is code-switching in Tamil and English, which "defies both the Tamil-only ideology in the public domains and institutions, and the English-only ideology from the ESL/TESOL pedagogical prescriptions from the West".

Apart from providing an insight as to how code-switching is used, this paper is also relevant because it discusses the stigma of code-switching, which can be seen as an inferior command of one or both languages. It discusses the challenges faced by ideological forces, where researchers are "working against the grain of dominant theories of the field". Therefore, they feel the constant need to justify and prove themselves. However, research has shown that in an educative setting, code-switching at least does no harm, and there are suggestions that it may even boost learning and understanding when measured in scores.

\bibliography{references}{}
\bibliographystyle{IEEEtran}

\end{document}
