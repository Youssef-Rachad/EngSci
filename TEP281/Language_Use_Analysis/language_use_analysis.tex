\documentclass[12pt]{article}
\usepackage{../../template}
\usepackage[margin=1in]{geometry}
\usepackage{listings}
\lstset{breaklines=true}
\lstset{basicstyle=\small\ttfamily}
\author{Daniel Chua}
\title{Language Use Analysis}
\begin{document}
\maketitle

\section{Introduction}

In this paper, I will mostly focus on \textit{my} own use of language. However, I will occasionally mention Matthew's own use of language, as his use of language demands my response. Similar to how different unities form a unity whose unique properties do not survive any division of said unity, I believe I cannot fully describe my use of language while keeping verbal and nonverbal language separate. Instead, I will attempt to analyse specific sentences and phrases, then my general use of language as a whole. (Examples of language use quoted from the transcript are in brackets)

For reference, the full transcript is included in Section \ref{transcript}.

\section{Specific Language Analysis}

I began by suggesting how certain concepts may be connected to each other, in response to Matthew's speech "let's see", which I took as an invitation to bring up new ideas. The phrase may also be used to start a sentence, but his placement of his fingers on his chins made me think he had not fully constructed anything yet. This led to my response. Since this was a casual context, and both of us could see the concepts I was pointing at, I simply used "this" and "these" instead of explicitly naming concepts.

Matthew responded by acknowledging my point (Fine you can say...), but then he brought forth his counter argument. This coupled with hesitation prompted me to think of ways to better explain my ideas, as seen in how I looked at the post-it note.

I then found a better connection out of a sudden, which was "no neutral words". I expressed this suddenness with a raised voice, noticeable gestures, and a look at Matthew, which was to ask for a response.

Matthew tried to construct a response, but he stuttered, as shown by his pauses, and his gesture that suggested thinking (hand on head). I tok this as an opportunity to further elaborate on a point, where I used more descriptive language and technical terms (demands a response, in an opposite way) instead of "filler" words.

Matthew explained his thoughts in more detail. I then interrupted when he pauses to think (hand on chin). I started by stating my stance (I don't), then I explained the details with the help of the concept map in front of us. As I was finding the right words, I revealed this thought process through words like um, and I looked far away, to avoid looking at anything that would distract me, as well as his gaze which would put pressure on me to respond. Once I have constructed my argument, I resume with technical terms again (meaning of language), with reference to the concept map, and I look at him to catch his attention and to show I was done thinking.

In Matthew's response, he looked at me, right before he asked for a clarification. Therefore, I looked back at him when making my response, to acknowledge his question. This was just an acknowledgement, however, as I had not fully thought out my point, and I paused while looking away. I then came up with a short answer.

In Matthew's response, I saw the use of "filler" words (But still, from) which did not mean anything by themselves. He also paused and made a spinning gesture with his hands. Therefore I started with my points. I point at myself a couple of times, where the point either referred to me, Daniel, or the concept of "me", which would be Matthew himself to Matthew, whom I would address as "you". Of course, this is needlessly complicated, and I am quite sure the nuances of this was not delivered. I then referred to the concept map and technical points when verbalising my thoughts. I also leaned forward to express my engagement. Then again, I stuttered, repeating my words and looked away. I tried to "force" myself to get back on track with hand gestures (H) to emphasise my points. When I started to lose track of what I was saying again, I adopted a different strategy, which was to ask Matthew for clarification (do you know what I mean). I then used my hand gestures to get back on track. Once I was confident in what I wanted to say, I again looked at Matthew to affirm that.

\section{General Language Analysis}

This will be a summary of general trends of language use I have noticed in the section above. I split my language use according to modes, loosely speaking, which are my "thinking" phase and my "speaking" phase.

In my "thinking" phase, I am constructing my argument, regardless if I am speaking or not. If I am not speaking, I stay silent, and there is a lack of gestures. If I am speaking, I tend to stutter, repeat myself, and use "filler" words, such as um, or words that don't really mean anything by themselves, such as "like", "but then", etc. I would also look away to avoid distractions, and to avoid looking at the other person, which would put me under additional pressure.

In my "speaking" phase, I have constructed my point, and I am trying to voice it out, regardless if I am speaking or not. If I am not speaking, I actively look out for opportunities to interrupt, based on how I interpreted the other party's use of language. I start by affirming that I know what I am saying. Verbally, this means I avoid "filler" words and use technical or revelant terms to get my ideas across. I would look at my partner in the eye to grab their attention, and to display confidence. I would also "strike" my hands to form a beat, or a rhythm of concepts.

As a side note, this justifies my choice to combine both verbal and nonverbal language, as my "thinking" and "speaking" phases cannot be completely characterised without both being present.

\appendix

\section{Transcript} \label{transcript}

\begin{lstlisting}

06:38 - 08:36

Notes: Unless if prepended by a name, actions [in these brackets] are performed by the speaker. M refers to Matthew, D refers to Daniel, B refers to both. = refers to an interruption by another speaker. P refers to the speaker pointing at the concept map, XP refers to X pointing at the concept map, and PX refers to the speaker pointing at. L1 refers to looking at the other speaker, and L2 refers to looking behind the camera. H refers to a downwards motion of the hand, similar to how one's arm goes down when lifting weights with a single hand.

M: Let's see [finger on chin]
D: I think this connects to these [BP]
M: Didn't we connects the role
M: Fine you can say everything connect to it, like, but I feel like this [P] is more... it shapes like how you literally see [looks around], or like can think about the objects around you [M picks up post-it, D looks at it], or is that like = the idea [finger quotes] of linguistic relativity right?
D: = [P vigorously] Oh, no neutral words! [looks at M]
M: That makes sense. Although I feel like this is kind of different [hand on head], I mean it is pretty close =
D: = It demands a response, I feel like it's related [M leans back], in an opposite way
M: I feel like [P] this is a little bit removed this as well, like um, the idea of like a response [raises hands as if holding a ball], kind of differs from like "guides a social reality". [P] But then again it's social, so then [hand on chin] = this is like framing differently
D: Yeah, I don't, I guess basically [P] it kinda feel that [L2] um the way you think in [L2], ties into the meaning of language [P], and this is that language [L1]... demands a response
M: [L1] It demands a response, like from another agent, you mean?
D: [L1] Yeah, [L2] but... but the response doesn't have to be verbal, obviously
M: But still, from [rotates index finger in the direction upwards, forwards, downwards, backwards/along horizontal axis] =
D: So like, if you're saying something [PD] to me [PD], or if you're saying it to yourself [PD], you're [PD], actively using [P] language, to demand some sort of like response [leans forwards, P], and I feel like this ties into, this ties into how, like, this [L2] ties into how, how the language [H] you use, the language you speak in [H], is whatever you think is important, [L2] do you know what I mean? [H] Demands something like, for this course, everytime you introduce a [H] reading, there is a [H] reading response. So this [L1], almost entirely [H]  dictates [H] how you're using language and meaning = , because [H]
M: [leans forwards] But you mean language and meaning, but not necessarily like social reality [P]

\end{lstlisting}
\end{document}
