\documentclass[12pt]{article}
\usepackage{../../template}
\author{niceguy}
\title{Lecture 9}
\begin{document}
\maketitle

\section{Definitions of Thermodynamic Quantities}

\begin{defn}[Entropy]
	$$S(E,V,N) = k\ln\Omega(E,V,N)$$
\end{defn}

\begin{defn}[Temperature]
	$$\frac{1}{T} = \left(\frac{\partial S(E,V,N)}{\partial E}\right)_{V,N}$$
\end{defn}

\begin{defn}[Pressure]
	$$\frac{p}{T} = \left(\frac{\partial S(E,V,N)}{\partial V}\right)_{E,N}$$
\end{defn}

\begin{defn}
	$$\frac{\mu}{T} = \left(\frac{\partial S(E,V,N)}{\partial N}\right)_{E,V}$$
\end{defn}

\section{Stirling's Approximation}

We try to approximate $\Omega$ given $q >> N$.
We know
$$\Omega(q,N) = \frac{(N-1+q)!}{(N-1)!q!} \approx \frac{(N+q)!}{N!q!}$$
Then
$$\ln\Omega \approx \ln(N+q)! - \ln N! - \ln q! \approx (N+q)\ln(N+q) - N\ln N - q\ln q$$
and so
$$\ln\Omega \approx N\ln(N+q) + q\ln(N+q) - N\ln N - q\ln q$$
Note that
$$\ln(N+q) = \ln\left(q\left(1+\frac{N}{q}\right)\right) = \ln q + \ln\left(1+\frac{N}{q}\right) \approx \ln q + \frac{N}{q}$$
Substituting and cancelling the terms,
$$\ln\Omega = \ln\left(\frac{qe}{N}\right)^N$$
Then putting
$$q = \frac{E}{\hbar\omega}$$
we get
$$\frac{1}{T} = \frac{kN}{E}$$
Now for $N >> q$, we have
$$\Omega(q,N) = \ln\left(\frac{Ne}{q}\right)^q$$
So
$$\frac{1}{T} = \frac{k}{\hbar\omega}\ln\frac{N\hbar\omega}{E} \Rightarrow e^{\frac{\hbar\omega}{kT}} = \frac{N\hbar\omega}{E}$$
or
$$\frac{E}{N} = \hbar\omega e^{-\frac{\hbar\omega}{kT}}$$

\begin{ex}
	Consider two containers $A$ and $B$ placed side by side, with $q_A = \frac{q}{2} + x$ and $q_B = \frac{q}{2} - x$. We want to find the most likely value of $x$. Obviously, it should be 0 by intuition/symmetry. But we can also solve for this.
	$$P(x) \propto \Omega_A\left(\frac{q}{2}+x,N\right)\Omega_B\left(\frac{q}{2}-x,N\right) = \left(\frac{e^2}{N^2}\right)^N\left(\frac{q}{2}-x\right)^N(\frac{q}{2}+x)^N = \left(\frac{e^2}{N^2}\right)^N\left(\left(\frac{q}{2}\right)^2-x^2\right)^N$$
	Now
	$$\frac{P(x)}{P(0)} = \left(1 - \left(\frac{2x}{q}\right)^2\right)^N$$
	Taking the logarithm,
	$$\ln\frac{P(x)}{P(0)} = N\ln\left(1-\left(\frac{2x}{q}\right)\right) \approx -N\left(\frac{2x}{q}\right)^2$$
	So
	$$P(x) \approx P(0) e^{-N\left(\frac{2x}{q}\right)^2}$$
	When $N$ is large, this is practically a delta function, and $x=0$ is the only value where $P$ is nonzero.
\end{ex}

\end{document}
