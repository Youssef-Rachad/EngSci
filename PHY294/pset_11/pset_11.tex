\documentclass[answers]{exam}
\usepackage{../../template}
\author{niceguy}
\title{Problem Set 11}
\begin{document}
\maketitle

\begin{questions}

\question{Consider a monatomic ideal gas that lives at a height $z$ above sea level, so each molecule has potential energy $mgz$ in addition to its kinetic energy.}

\begin{parts}
    \part{Show that the chemical potential is the same as if the gas were at sea level, plus an additional term $mgz$}

    \begin{solution}
        The new potential energy is given by
        $$U' = U + mgzN$$
        Then
        $$\mu' = \frac{dU'}{dN} = \frac{dU}{dN} + \frac{d}{dN} mgzN = \mu + mgz$$
    \end{solution}

    \part{Suppose you have two chunks of helium gas, one at sea level and one at height $z$, each having the same temperature and volume. Assuming that they are in diffusive equilibrium, show that the number of molecules in the higher chunk is
    $$N(z) = N(0)e^{-mgz/kT}$$
}

    \begin{solution}
        \begin{align*}
            -kT\ln[\frac{V}{N_0}\left(\frac{2\pi mkT}{h^2}\right)^3/2] &= -kT\ln[\frac{V}{N_B}\left(\frac{2\pi mkT}{h^2}\right)^{3/2}] + mgz \\
            -kT\ln\frac{1}{N_0} &= -kT\ln\frac{1}{N_B} + mgz \\
            \ln\frac{N_0}{N_B} &= \frac{mgz}{kT} \\
            N(z) &= N_0 e^{-\frac{mgz}{kT}}
        \end{align*}
    \end{solution}
\end{parts}
\end{questions}
\end{document}
