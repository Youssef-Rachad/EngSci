\documentclass[12pt]{article}
\usepackage{../../template}
\author{niceguy}
\title{Lecture 14}
\usepackage{fontspec}
\setmainfont{Comic Sans MS}
\begin{document}
\maketitle

At equilibrium, a system of two gases would tend to the same pressure and temperature, where entropy is maximised. This means
$$\frac{\partial S_1}{\partial N_1} = \frac{\partial S_2}{\partial N_2}$$
Now we define this as
$$-\frac{\mu}{T} = \frac{\partial S}{\partial N}$$
where $\mu$ is the chemical potential, with units of energy. Using our favourite equation
$$S = kN\left[\ln\left(\frac{V}{N}\left(\frac{U}{3N}\right)^{\frac{3}{2}}\left(\frac{4\pi m}{h^2}\right)^{\frac{3}{2}}\right) + \frac{5}{2}\right]$$
Then differentiating and skipping the steps
\begin{align*}
    \frac{\mu}{T} &= -k\left[\ln\left(\frac{V}{N}\left(\frac{U}{3N}\right)^{\frac{3}{2}}\left(\frac{4\pi m}{h^2}\right)^{\frac{3}{2}}\right) + \frac{5}{2}\right] + kN\frac{5}{2N} \\
    \mu &= -kT\ln\left[\frac{V}{N}\left(\frac{1}{2}kT\right)^{\frac{3}{2}}\left(\frac{4\pi m}{h^2}\right)^{\frac{3}{2}}\right] \\
        &= -kT\left[\ln\left(\frac{1}{n} \frac{1}{\lambda_{th}^3}\right) + C\right]
\end{align*}
where $C$ is a constant and $n$ is the number density. With the negative sign, $\mu$ increases with $n$, which makes sense.

\section{Saha Equation}
\end{document}
