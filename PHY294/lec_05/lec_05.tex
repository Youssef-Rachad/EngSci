\documentclass[12pt]{article}
\usepackage{../../template}
\author{niceguy}
\title{Lecture 5}
\begin{document}
\maketitle

\section{Adiabetic Process}

$$\Delta U = Q - p\Delta V = -p\Delta V$$

We also use $U = \frac{f}{2}NkT$, where $f=3$ for the single atomic case, and $f=7$ for the diatomic case. Combining gives
$$-p\Delta V = \frac{f}{2}Nk\Delta T$$
Now from the ideal gas law,
$$\Delta(pV) = Nk\Delta T$$
Using the product rule,
\begin{align*}
	-p\Delta V &= \frac{f}{2}(p\Delta V + V\Delta p) \\
	-\left(\frac{f}{2}+1\right)p\Delta V &= \frac{f}{2}V\Delta p \\
	-\frac{\frac{f}{2}+1}{\frac{f}{2}} \frac{\Delta V}{V} &= \frac{\Delta p}{p} \\
	-\frac{\frac{f}{2}+1}{\frac{f}{2}} d\ln V &= d\ln p \\
	-\left(1+\frac{2}{f}\right)\ln\frac{V_2}{V_1} &= \ln\frac{p_2}{p_1} \\
	\left(\frac{V_1}{V_2}\right)^{1+\frac{2}{f}} &= \frac{p_2}{p_1} \\
	V_1^{1+\frac{2}{f}}p_1 &= V_2^{1+\frac{2}{f}}p_2
\end{align*}

Thus $pV^{1+\frac{2}{f}}$ is held constant. \\
Similarly, we have isohoric processes, with a constant $V$, and isobaric processes, with a constant $p$.

\section{Heat Capacity}

For constant volume,
$$C_V = \left(\frac{\partial U}{\partial T}\right)_{V,N}$$
The first law of thermodynamics simplifies to
$$\Delta U = Q$$
The general definition is then
$$C_{\text{condition}} = \left(\frac{Q}{\Delta T}\right)_{\text{condition}}$$

For constant pressure,
$$C_p = \left(\frac{Q}{\Delta T}\right)_{p,N} = \left(\frac{\Delta U + p\Delta V}{\Delta T}\right)_{p,N} = \left(\frac{\partial U}{\partial T}\right)_{p,N} + \left(\frac{\partial V}{\partial T}\right)_{p,N}$$
Hence $C_p > C_V$ always holds. For ideal gases,
$$U = \frac{f}{2}NkT$$
Substituting, this gives
$$\left(\frac{\partial U}{\partial T}\right)_{p,N} = \left(\frac{\partial U}{\partial T}\right)_{V,N} = \frac{f}{2}Nk$$
and from the ideal gas law,
$$\left(\frac{\partial V}{\partial T}\right)_{p,N} = \frac{Nk}{p}$$

\section{Main Postulate}
Consider a closed system with fixed energy. Then all accessible microstates of the system are equally likely.
\end{document}
