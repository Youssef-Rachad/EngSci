\documentclass[12pt]{article}
\usepackage{../../template}
\author{niceguy}
\title{Lecture 17}
\begin{document}
\maketitle

\section{Boltzmann Distribution}

In a thermostat ($T,V,N$), the probability of a microstate is given by

$$P(\text{microstate}) = \frac{1}{z} e^{-\frac{E}{kT}}$$

And the probability of a system with energy $E$ is

$$P(\text{energy}) = P(\text{microstate})\Omega(E) = \frac{1}{z} e^{-\frac{E-TS(E)}{kT}}$$

The most likely energy is where the numerator of the exponent. Defining the free energy

$$F = E-TS(E)$$

we can simply look for the minimum free energy. Note that

$$dF = -SdT - pdV + \mu dN$$

The proof is left as an exercise for the reader. Now given $F$, we see for a low $T$, $F$ is dominated by the $E$ term, so the lowest energy arrangement is the most likely. Similarly, for a high $T$, $F$ is dominated by the $-TS$ term, so the arrangement where $S$ is maximised is the most likely. This is the \textbf{order disorder} transition.

\section{Working with z}

\begin{align*}
    z &= \sum_{q_1=0}^\infty \sum_{q_2=0}^\infty \dots \sum_{q_N=0}^\infty e^{-\frac{\hbar\omega}{kT}q_1} e^{-\frac{\hbar\omega}{kT}q_2} \times \dots \times e^{-\frac{\hbar\omega}{kT}q_N} \\
      &= \left(\sum_{q_1=0}^\infty e^{-\frac{\hbar\omega}{kT}q_1}\right)\left(\sum_{q_2=0}^\infty e^{-\frac{\hbar\omega}{kT}q_2}\right)\times\dots\times\left(\sum_{q_N=0}^\infty e^{-\frac{\hbar\omega}{kT}q_N}\right) \\
      &= (z_1)^N
\end{align*}

Now $z_1$ itself is a geometric series, so
$$z = (z_1)^N = \left(\frac{1}{1-e^{-\hbar\omega/kT}}\right)^N$$

We attempt to find the average energy. Now letting $\beta = \frac{1}{kT}$,

\begin{align*}
E_{\text{avg}} &= \sum_{\text{microstate}} EP \\
               &= \sum_{\text{microstate}} -\frac{1}{z} \frac{\partial}{\partial\beta} e^{-\beta E} \\
               &= -\frac{1}{z} \frac{\partial}{\partial\beta}\left(\sum e^{-\beta E}\right) \\
               &= -\frac{1}{z} \frac{\partial z}{\partial\beta} \\
               &= -\frac{\partial}{\partial\beta}\ln z \\
               &= N\frac{\partial}{\partial\beta}\ln\left(1-e^{-\beta\hbar\omega}\right) \\
               &= \frac{N\hbar\omega}{e^{\frac{\hbar\omega}{kT}}-1}
\end{align*}

Now at high $T$, we approximate $e^x-1 \approx x$, so that gives us
$$\overline E \approx NkT$$
\end{document}
