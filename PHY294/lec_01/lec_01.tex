\documentclass[12pt]{article}
\usepackage{../../template}
\title{Lecture 1}
\author{niceguy}
\begin{document}
\maketitle

AN: hoepfully this goes better than the first half of the course.

\section{Thermal Physics}

These are systems with many particles, to the order of $10^{23}$. Hence, we approximate that as infinitely many particles. We want to learn the "universal behaviour" which obey laws independent of microscopic detail. For example, the \textbf{Boltzmann Distribution} in 3 dimensions is

$$v^2dve^{-kv^2}dv$$

We have thermodynamics and statistical mechanics, where the latter is a microscopic "derivation" of the former. \\

Consider $5\unit{cm^3}$ of olive oil which can be spread over at most $2000\unit{m^2}$. This gives the size of molecules, which is $2.5\times10^{-9}\unit{m}$. The number of molecules is to the order of $10^{21}$. \\

To fully describe a system of $N$ particles, we need the initial positions and velocities of each particle, which is $6N$ variables. This is hopeless to compute and useless to interpret. \\
A thermodynamic equilibrium is reached when

\begin{enumerate}
	\item Uniformity throughout V
	\item Pressure and Temperature will have constant values
	\item There are no macroscopic fluxes
\end{enumerate}

\section{Ideal Classical Gas}

These gases are formed by many classical point like particles that follow the ideal gas law in general.

$$pV = NkT$$
\end{document}
