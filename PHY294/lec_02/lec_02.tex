\documentclass[12pt]{article}
\usepackage{../../template}
\author{niceguy}
\title{Lecture 2}
\begin{document}
\maketitle

\section{Ideal Gas}

When plotting pressure against temperature for a gas, the experimental data shows that all such lines meet at a pressure of 0 and temperature of -273$^\circ$C. This is called the \textbf{absolute zero}.

\subsection{Derivation}

We assume that all molecules move with the same typical average velocity, and the velocity in any given physical olume is isotropically distributed. Then by symmetry,

$$\overline{v}_x = \frac{1}{N} \sum_i v_{x_i} = 0$$

And

$$\overline{v_x^2} = \frac{1}{N} \sum_i v^2_{x_i} = \overline{v}^2 \neq 0$$

Letting $L$ be the box length, and $\Delta t$ such that
$$L = \overline{v}\Delta t$$

The number of molecules hitting the wall in time $\Delta t$ is half of the molecules in the volume, i.e.
$$\frac{1}{2}\frac{N}{V}LA = \frac{1}{2} \frac{N}{V} \overline{v}\Delta tA$$

then the change in momentum per unit area of the wall in the time interval $\Delta t$ is
$$2m\overline{v} \times \frac{N}{V} \frac{1}{2} \overline{v}\Delta tA$$

Force is but the change in momentum over time, and pressure is force per area, which gives
$$p = m\overline{v}^2\frac{N}{V}$$

Comparing this with the ideal gas law,
$$kT = m\overline{v}^2 = \frac{1}{3} m\overline{\vec{v}^2}$$
Rearringing, we get the kinetic energy
$$\frac{1}{2}m\overline{\vec{v}^2} = \frac{3}{2}kT$$

This means temperature is a measure of average kinetic energy! \\
The root mean squared veolcity is
$$v_{\text{rms}} = \sqrt{\overline{\vec{v}^2}} = \sqrt{\frac{3kT}{m}}$$

\section{Equipartition Theorem}

For classical gasses: At thermodynamic equilibrium of a classical gas at source given $T$ the average energy of

\begin{itemize}
	\item a translational degree of freedom is $\frac{kT}{2}$
	\item a rotational degree of freedom is $\frac{kT}{2}$
	\item a vibrational degree of freedom $kT$
\end{itemize}

\begin{ex}
	For a system of 2 atoms, there are 3 degrees of freedom for translation, 2 for rotation, and 1 for vibration.
\end{ex}

\begin{ex}
	For N$_2$, average energy is
	$$3\times \frac{kT}{2} + 2\times \frac{kT}{2} + kT = \frac{7kT}{2}$$
\end{ex}

\end{document}
