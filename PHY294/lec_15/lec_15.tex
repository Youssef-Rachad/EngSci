\documentclass[12pt]{article}
\usepackage{../../template}
\author{niceguy}
\title{Lecture 15}
\begin{document}
\maketitle

\section{Thermodynamic Identity for Entropy}

We call entropy $S$ the thermodynamic potential (but entropy "wants" to increase while potential "wants" to decrease). If we take the total differential,

\begin{equation}\label{siden}
    dS = \frac{\partial S}{\partial U}dU + \frac{\partial S}{\partial V}dV + \frac{\partial S}{\partial N}dN = \frac{1}{T} dU + \frac{p}{T} dV - \frac{\mu}{T} dN
\end{equation}

This holds under quasistatic infinitesimal changes of $U,V,N$. This needs to be quasistatic or else we do not have good definitions for $T, p$, etc, without equilibrium.

\section{Thermodynamic Identity for Energy}

Rearranging $S=k\ln\Omega$ to put $U$ as the subject, we obtain
$$e^{\frac{S}{kN}} = U^{\frac{3}{2}}f(N,V)$$
where $f$ is a constant function of $N$ and $V$. Multiplying Equation \ref{siden} by $T$,
$$dU = TdS - pdV + \mu dN$$
Note also $U$ is a function of $S,V,N$, so
$$dU = \frac{\partial U}{\partial S}dS + \frac{\partial U}{\partial V}dV + \frac{\partial U}{\partial N}dN$$
Then comparing like terms,
$$T = \frac{\partial U}{\partial S}, p = -\frac{\partial U}{\partial V}, \mu = \frac{\partial U}{\partial N}$$

Now extreme points are found at $dS=0$, but we do not know if it is a maximum, minimum, or neither. We also require $C_V > 0, \frac{\partial p}{\partial V} < 0, \frac{\partial \mu}{\partial N} > 0$. Heat capacity has to be positive, pressure and volume always go in the opposite directions, and chemical potential increases with number density (volue is held constant, number of particles increase).

\end{document}
