\documentclass[12pt]{article}
\usepackage{../../template}
\author{niceguy}
\title{Lecture 11}
\begin{document}
\maketitle

\section{Fundamental Theorem for Line Integrals}
If $\vec{F}$ is a gradient of a scalar function $f$, then
$$\int_a^b \vec{F} \cdot d\vec{r} = f(\vec{r_b}) - f(\vec{r_a})$$. If $C = \bigcup_i C_i$ and $C' = \bigcup_i C_i'$ where both share the same endpoints, we can see that
$$\int_C \vec{F} \cdot d\vec{r} = \int_{C'} \vec{f} \cdot d\vec{r}$$
In fact, if the curve starts and ends at the same point, we call this a \emph{closed} integral where
$$\oint \vec{F} \cdot d\vec{r} \equiv 0$$
Conversely, if this is true for any closed curve $C$, we know that it must be path independent. (Consider two paths $C_1$ and $C_2$ with the same endpoints. Then their difference is a closed curve which is zero, implying they are equal). \\
To sum up, the following statements are equal
\begin{enumerate}
	\item $\vec{F}$ is conservative
	\item the integral of $\vec{F} \cdot d\vec{r}$ is path independent
	\item $\oint \vec{F} \cdot d\vec{r} \equiv 0$
\end{enumerate}

Consider $\vec{F} = P(x,y)\hat{i} + Q(x,y)\hat{j}$. If we assume $P(x,y) = \frac{\partial f}{\partial x}$ and $Q(x,y) = \frac{\partial f}{\partial y}$, we have
$$\frac{\partial P}{y} = \frac{\partial Q}{x}$$
assuming mixed partials commute. Hence the above equation can be used to verify if $\vec{F}$ is conservative or not (proof not given).

\begin{ex}
	$$\vec{F}(x,y) = y\hat{i} - x\hat{j}$$
	$\vec{F}$ is not conservative, as the mixed partials are not equal.
\end{ex}

\begin{ex}
	$$\vec{f}(x,y) = y\hat{i} + x\hat{j}$$
	$\vec{F}$ is conservative, as the mixed partials are equal. Integrating $P$ and $Q$ and using $g(y)$ and $g(x)$ as "constant" functions, we get $f = xy + C$ where $g(x) = g(y) = C$ by comparison. 
\end{ex}

\section{Green's Theorem}

\subsection{Terminology}

\begin{itemize}
	\item Simple curve: a curve that does not intersect itself except at endpoints \\
	\item Orientation: orientation is positive when curve goes counterclockwise, vice versa
\end{itemize}

\subsection{Theorem and Proof}

\begin{thm}
	Let $ $ be a positively oriented, piecewise-smooth, simple closed curve in the plane, and let $R$ be the region bounded by $C$. If $P(x,y)$ and $Q(x,y)$ are continuous and have continuous first partial derivatives throughout the region $R$, then
	$$\oint_C P(x,y)dx + Q(x,y)dy = \iint_R \left(\frac{\partial Q}{\partial x} - \frac{\partial P}{\partial y}\right) dxdy$$
\end{thm}

Proof: \\
We can express the region as $C = C_1 \cup C_2$ or $C = C_3 \cup C_4$ where
$$R = \{(x,y)| a \leq x \leq b, y_1(x) \leq y \leq y_2(x)\}$$
and
$$R = \{(x,y)| x_1(y) \leq x \leq x_2(y), c \leq y \leq d\}$$
where $a, b$ are the left and right endpoints, and $c, d$ are the top and bottom endpoints. We can write
\begin{align*}
	\oint_C P(x,y)dx &= \int_{C_1}P(x,y)dx + \int_{C_2}P(x,y)dx \\
			 &= \int_a^bP(x,y_1(x))dx + \int_b^aP(x,y_2(x))dx \\
			 &= -\int_a^bP(x,y_2(x))-P(x,y_1(x))dx \\
			 &= -\int_a^b P(x,y) \Big |_{y=y_1(x)}^{y=y_2(x)}dx \\
			 &= -\int_a^b \int_{y_1(x)}^{y_2(x)} \frac{\partial P}{\partial y} dydx \\
\end{align*}
Similarly,
\begin{align*}
	\oint_C Q(x,y)dy &= \int_{C_3}Q(x,y)dy + \int_{C_4}Q(x,y)dy \\
			 &= \int_d^cQ(x_1(y),y)dy + \int_c^dQ(x_2(y),y)dy \\
			 &= \int_c^d Q(x_2(y),y) - Q(x_1(y),y)dy \\
			 &= \int_c^d Q(x,y) \Big |_{x=x_1(y)}^{x=x_2(y)} dy \\
			 &= \int_c^d \int_{x_1(y)}^{x_2(y)} \frac{\partial Q}{\partial x} dxdy \\
\end{align*}
Summing both gives us
$$\oint Pdx + Qdy = \iint_R \left(\frac{\partial Q}{\partial x} - \frac{\partial P}{\partial y}\right) dxdy$$

\subsection{Examples}

\begin{ex}
	Verify Green's Theorem for the integral $\oint_Cydx - xdy$ where $C$ is the circle $C:x^2+y^2=1$ traversed counterclockwise. \\
	The left hand side gives
	\begin{align*}
		I &= \oint \vec{F} \cdot d\vec{r} \\
		  &= \int_0^{2\pi} (\sin t\hat{i}-\cos t\hat{j})\cdot(-\sin t\hat{i}+\cos t\hat{j})dt \\
		  &= \int_0^{2\pi} -\sin^2 t - \cos^2 t dt \\
		  &= -2\pi
	\end{align*}
	The right had side gives
	\begin{align*}
		I &= \iint_R\left(\frac{\partial Q}{\partial x} - \frac{\partial P}{\partial y}\right) dA \\
		  &= \iint_R -2dA \\
		  &= -2\pi
	\end{align*}
	as the area of a unit circle is $\pi$. 
\end{ex}

\begin{ex}
	$$I = \oint_C \left(4-e^{\sqrt{x}}\right)dx + \left(\sin y + 3x^2\right)dy$$
	where $C$ goes counter clockwise around the disk bounded by $a \leq r \leq b$ in the first quadrant.
	\begin{align*}
		\oint_C \left(4-e^{\sqrt{x}}\right)dx + \left(\sin y + 3x^2\right)dy &= \iint_R 6x dA \\
										     &= \int_0^{\frac{\pi}{2}} \int_a^b 6r\cos \theta rdrd\theta \\
										     &= \int_0^{\frac{\pi}{2}} 2(b^3-a^3) \cos\theta d\theta \\
										     &= 2(b^3-a^3)
	\end{align*}
\end{ex}

\section{Parametric Surfaces and their Surface Areas}

Just as how a curve can be parametrised by one variable, a surface can be parametrised by two variables, where
$$\vec{r}(u,v) = x(u,v)\hat{i} + y(u,v)\hat{j} + z(u,v)\hat{k}$$
The simplest way to do this is by $S: z = f(x,y)$ where $u=x$ and $v=y$.

\begin{ex}
	Parametrise an upper hemisphere given by the equation $x^2 + y^2 + z^2 = a^2$. \\
	Rearranging, we have
	$$z = \sqrt{a^2-x^2-y^2}$$
	so we have
	$$\vec{r}(u,v) = u\hat{i} + v\hat{j} + \sqrt{a^2-u^2-v^2}\hat{k}$$
	We can also use spherical coordinates which gives
	$$\vec{r}(u,v) = a\vec{r} + u\hat{\theta} + v\hat{\phi}$$
	where $u \in [0,2\pi], v \in \left[0,\frac{\pi}{2}\right]$.
\end{ex}

\section{Tangent Planes}

Let $S$ be a surface parametrised by the differentiable vector function
$$\vec{r}(u,v) = (u,v)\hat{i} + y(u,v)\hat{j} + z(u,v)\hat{k}$$
where $(u,v) \in D$. \\
Then the tangent plane at $(u_0,v_0)$ is spanned by
$$\vec{r_v}(u) = \frac{\partial \vec{r}}{\partial v} \Big |_{u_0,v_0}$$
and
$$\vec{r_u}(v) = \frac{\partial \vec{r}}{\partial u} \Big |_{u_0,v_0}$$

\begin{defn}
	A surface is \emph{smooth} if for every point,
	$$\vec{r_v}(u) \times \vec{r_u}(v) \neq 0$$
\end{defn}
\end{document}
