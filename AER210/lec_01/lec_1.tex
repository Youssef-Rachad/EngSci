\documentclass[12pt]{article}
\usepackage{../../template}
\title{Lecture 1}
\author{niceguy}
\begin{document}
\maketitle
\section{Integrals Involving a Parameter}

\begin{ex}
 $$\int_0^1 Cx^3 dx$$
 $C$: constant, $x$: variable
\end{ex}

\begin{ex}
$$ \int_0^1 Cx_3 dx = C\frac{x^4}{4}\Big |_0^1 = \frac{1}{4}C$$
Result contains $C$
\end{ex}

\begin{ex}
 $$\int_a^b f(x,y)dx = g(y)$$
\end{ex}

\begin{defn}
 A variable which is kept constant during an integration is called a parameter.
\end{defn}

\begin{ex}
 $$\int_0^1 x^3ydx = y\int_0^1 x^3 dx = \frac{y}{4}$$
 Where $y$ is the parameter.
\end{ex}

\subsection{Integrated Integrals (Integral of an Integral)}

$z = f(x,y)$ where $x \in [a,b], y \in [c,d]$ \\
Assume $f(x,y) \geq 0$ \\
The area of a vertical slice at a given $x$ is
$$\int_c^d f(x,y)dy = A(x)$$
The volume of said slice is
$$\Delta V(x) = A(x) \Delta x = \left( \int_c^d f(x,y)dy \right) \Delta x$$
Consider a partition of $[a,b]$, and we have
$$V \approx \sum_{i=1}^N \Delta V_i = \sum_{i=1}^N A(_i)\Delta x_i$$
As $\Delta x_i \rightarrow 0$, the Riemann sum gives us the integral
$$V = \int_a^b A(x)dx = \int_a^b \int_c^d f(x,y)dydx$$
The same can be done in the reverse order, i.e. with $A'(y)$ and $V = \int_c^d A(y) dy$

\begin{thm}
 Fubini's Theorem. \\
 $$\int_a^b \int_c^d f(x,y)dydx = \int_c^d \int_a^b f(x,y) dxdy$$
 Proof: trust me bro
\end{thm}

\begin{ex}
 Find the volume under the surface $z = x^2y, x \in [1,3], y \in [0,1]$ \\
Forming the integral by first integrating with respect to $y$
\begin{align*}
 V &= \int_1^3 \int_0^1 x^2y dydx \\
 &= \int_1^3 \frac{x^2y^2}{2} \Big |_0^1 dx \\
 &= \int_1^3 \frac{x^2}{2} dx \\
 &= \frac{x^3}{6} \Big |_1^3 \\
 &= \frac{13}{3} \\
\end{align*}
Forming the integral by first integrating with respect to $x$
\begin{align*}
 V &= \int_0^1 \int_1^3 x^2y dxdy \\
&= \int_0^1 \frac{x^3y}{3} \Big |_1^3 dy \\
&= \int_0^1 \frac{26y}{3} dy \\
&= \frac{26y^2}{6} \Big |_0^1 \\
&= \frac{13}{3} \\
\end{align*}
\end{ex}

\begin{ex}
 Evaluate the double integral of $f(x,y) = x - 3y^2$ over region $R = \{(x,y)| 0 \leq x \leq 2, 1 \leq y \leq 2\}$
 $$\int_0^2 \int_1^2 (x-3y^2) dydx = \int_0^2 (xy - y^3) \Big |_1^2 dx = \int_0^2 x - 7 dx = \left(\frac{x^2}{2} - 7x\right) \Big |_0^2 = -12$$
\end{ex}

In the special case where $f(x,y) = g(x)h(y)$, then
$$\int_c^d \int_a^b f(x,y) dxdy = \int_c^d \int_a^b g(x)h(y)dxdy = \int_c^d h(y) \int_a^ g(x)dx dy = \int_a^b g(x)dx \int_c^d h(y)dy$$

\begin{ex}
$$ \int_0^{\frac{\pi}{2}} \int_0^{\frac{\pi}{2}} \sin x \cos y dxdy = \int_0^{\frac{\pi}{2}} \sin x dx \int_0^{\frac{\pi}{2}} \cos y dy = -\cos x \Big |_0^{\frac{\pi}{2}} + \sin y \Big |_0^{\frac{\pi}{2}} = 2$$
\end{ex}

\subsection{Double Integrals over General Regions}

Type 1 Region: $R = \{(x,y)| a \leq x \leq b, g_1(x) \leq y \leq g_2(x)\}$ \\
If $f(x,y) \geq 0$ on a type 1 region,
$$A(x) = \int_{g_1(x)}^{g_2(x)} f(x,y) dy$$
and similarly, we have
$$V = \int_a^b \int_{g_1(x)}^{g_2(x)} f(x,y) dydx$$

Type 2 Region: $R = \{(x,y)| c \leq y \leq d, h_1(y) \leq x \leq h_2(y)\}$ \\
Similarly,
$$V = \int_c^d \int_{h_1(y)}^{h_2(y)} f(x,y) dxdy$$

\begin{ex}
 Find the volume of the solid that lies under the surface $z = x^2 + y^2$ and above the region $R$ in the $xy$ plane. The region is bounded by the straight line $y = 2x$ and the parabola $y = x^2$.
 
 Integrating with respect to $y$ first,
\begin{align*}
 V &= \int_0^2 \int_{x^2}^{2x} x^2 + y^2 dydx \\
 &= \int_0^2 x^2y + \frac{y^3}{3} \Big |_{x^2}^{2x} dx \\
 &= \int_0^2 2x^3 - x^4 + \frac{8x^3}{3} - \frac{x^6}{3} dx \\
 &= \frac{x^4}{2} - \frac{x^5}{5} + \frac{2x^4}{3} - \frac{x^7}{21} \Big |_0^2 \\
 &= 8 - \frac{32}{5} + \frac{32}{3} - \frac{128}{21} \\
 &= \frac{216}{35} \\
\end{align*}

Integrating with respect to $x$ first,
\begin{align*}
 V &= \int_0^4 \int_{\frac{y}{2}}^{\sqrt{y}} x^2 + y^2 dxdy \\
 &= \int_0^4 \frac{x^3}{3} + xy^2 \Big |_{\frac{y}{2}}^{\sqrt{y}} dx \\
 &= \int_0^4 \frac{y^{\frac{3}{2}}}{3} + y^{\frac{5}{2}} - \frac{13}{24} y^3 dy \\
 &= \frac{216}{35} \\
\end{align*}
\end{ex}

It is sometimes easier to integration with respect to one variable over the other. 

\begin{ex}
 Integrate the surface given by $z = e^{x^2}$ over the region between $y = x$ and $y = 0$ for $x \in [0,1]$. \\
 
 If we first integrate with respect to $x$, this results in an integral which has no elementary antiderivative (though it can still be evaluated).
 
\begin{align*}
 V &= \int_0^1\int_y^1 e^{x^2}dx dy \\
\end{align*}

If we first integrate with respect to $y$,

\begin{align*}
 V &= \int_0^1 \int_0^x e^{x^2} dydx \\
 &= \int_0^1 ye^{x^2} \Big |_0^x dx \\
 &= \int_0^1 xe^{x^2} dx \\
 &= \frac{1}{2} e^{x^2} \Big |_0^1 \\
 &= \frac{e - 1}{2} \\
\end{align*}
\end{ex}
\end{document}  
