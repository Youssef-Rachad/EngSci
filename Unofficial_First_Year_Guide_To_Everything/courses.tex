Find past papers of courses in \url{courses.skule.ca}.

\section{CIV102: Structures and Materials}

Don't worry, those courses are mostly: "here's an equation, know where to use them". Nobody comes in knowing CIV, unless if you're a nerd. Just make sure to start your project and assignments as early as possible. They are also really generous with part marks, so don't give up just becausae you can't get the final result!

\section{ESC101: Praxis I}

THERE IS NO PREPARING FOR PRAXIS! You are either good at it, or bad at it. I would assume why most people don't like this course is because:

\begin{enumerate}
    \item Unbalanced return on effort spent
    \item Grades are very subjective
    \item There's lots of writing
\end{enumerate}

\section{ESC103: Engineering Mathematics and Computation}

There are two parts to this course. The first part is linear algebra, and covers vectors, matrices, dot/cross products, eigenvectors and eigenvalues, etc. The second part covers numeric methods such as Euler's method and the improved Euler's method. In the first part, you do questions in tutorials, then you have MATLAB labs in the second part. If you did vectors in high school, the first part should be easy, likewise for MATLAB.

\section{ESC180: Intro to Computer Programming}

Programming in python. Your whole experience in computer will be split into:

\begin{itemize}
    \item I know programming: you can basically skip lectures and review for one day before midterm and be fine
    \item What is programming???: you might suffer in this course, but worry not, this course actually moves quite slow... you will catch up
\end{itemize}

niceguy would like to add that he started coding in UTEA, less than 3 months before school started, and everything turned out fine. He's also met someone who learnt coding in first year, and proceeded to win in hackathons and is now doing his PEY at Intel.

\section{ESC194: Calculus I}

Calculus starts from basically defining what numbers are, though you do need basic trigonometry, e.g. the functions $\sin, \cos, \tan$, and how to evaluate them. You are introduced to epsilon-delta proofs which should be new to everyone

$$\forall \epsilon > 0 \exists \delta > 0 \text{ s.t. } 0 < |x-c| < \delta \Rightarrow |f(x)-L| < \epsilon$$

Other than that, it helps if you did calculus in high school, but the questions might be trickier.

\section{PHY180: Classical Mechanics}

Physics is basic, you can probably pass the exam if you go into it like... right now?

\section{ESC102: Praxis II}

The teamwork strategies are useful if you want to implement it. Learning to write an RFP is also good. However, the organisation was horrible, and everything I learnt was because I worked with a group and not because of Praxis, i.e. I would learn the same stuff in a design team.

\section{ESC190: Computer Algorithms and Data Structures}

Similar to ESC180, this is an introductory course to programming in C. You learn about data structures and algorithms. Some find this much harder than ESC180, especially if they have experience in Python. As a first-time coder, I found this course challenging but not unreasonable.
\section{ESC195: Calculus II}

The course starts with integrals and convergence. You learn Taylor series and Fourier series which have many engineering applications. You also learn a bit of vector calculus.

\section{MAT185: Linear Algebra}

Linear algebra is new to everyone, nothing can really prepare you for it. Maybe knowing vectors will help? Try to get a sense of how matrix multiplication work in terms of "combination of rows/columns" instead of a literal: $a_1\times b_1 + a_2\times b_2\dots$ sense.

At the moment of writing, ChatGPT is still pretty bad with linear algebra, so don't use it for your homework or exams.

You can also drop this course and do MAT223 instead, the ArtSci (Faculty of Arts and Science) equivalent. Again, please check with the EngSci Office before doing that. This EngSci upper year would also like to kindly inform you that the course is garbage. It is boring and covers no new content, so you should not take it unless if it's for free.

\section{ECE159: Electric Circuits}

Circuit is fine. High school stuff (even including AP/IB) probably only covers the first 2-3 weeks maximum.

\section{MSE160: Molecules \& Materials}

Don't worry, those courses are mostly: "here's an equation, know where to use them" (stress/strain, bending moments, density...)
