\section{Internationals}

I will add stuff later (read: never), but for now, check out the \href{https://internationalexperience.utoronto.ca/}{Centre for International Experience} website! Personally, I think the \href{https://internationalexperience.utoronto.ca/international-student-services/resource-and-information-hub/ise-roadmap/}{UofT Roadmap} is a good resource that summarises what you have to do from pre-arrival (what documents to get, what things to buy) to transitioning into university and all the way to transitioning out upon graduation!

\section{T-Card}

Your T-Card is your Student ID in UofT. You need it for identification purposes for midterms and exams, and to access University facilities, such as libraries, ECF Labs, the Common Room, Sports facilities, etc. \href{https://tcard.utoronto.ca/get-your-utorid-tcard/}{Get} yours as soon as possible.

\section{Food}

\subsection{If you have a meal plan}

Chestnut, New College

\subsection{Food Trucks}

The Green, Blue, and Pink food trucks all pretty good. 

\begin{itemize}
    \item Green: Shawarma wraps, pretty good
    \item Blue: Poutines (Pretty large) (In case you don't know what poutines are: Cheese on fries with hot gravy to "melt" the cheese). They have bacon poutines/chicken poutines/other...
    \item Pink: A very "Chinese" food truck, mostly chicken. They have popcorn chicken/burgers... Takoyaki... and rice
\end{itemize}

\subsection{Spadina}

There are a lot of good and cheap Chinese food here, as well as grocery stores. There is a Burger King and Popeyes nearby.

\subsection{College}

%add more stuff

\subsection{Coffee}

There is a Second Cup in Myhal second floor. There is also a Starbucks on College opposite to St. George.

\subsection{Groceries}

Hua Sheng (華盛) on Spadina, Lucky Moose (金牛城) on Dundas.
