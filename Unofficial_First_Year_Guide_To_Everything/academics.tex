This is where you can find general academic advice that applies to courses in general.

\section{Hardest Time of the Year}

\subsection{First Semester}
\begin{itemize}
    \item Pendulum report time, lots of data collection lots of writing
    \item First midterm (for us it's praxis), lots of "extra" preparation
    \item CIV bridge (build them early, and you get to watch other people suffer while you sleep :$>$)
    \item Most Praxis stuff
\end{itemize}

\subsection{Second Semester}
\begin{itemize}
    \item RFP (Praxis)
    \item Beta (Praxis)
    \item Showcase (Praxis)
    \item Handbook/Portfolio (Praxis)
    \item Yes only Praxis
\end{itemize}

\section{Is EngSci stressful/hard?}

\subsection{Andy}

It's pretty ok overall, same experience as in high school. Actually most days I am more chill cuz lectures are at 11am-1pm for 2nd semester so I can sleep in... Assignments are fine, course contents take some effort to understand, but once you get it it's pretty chill. I did AP+IB in high school. And I know programming (quite a lot of programming) before uni...

\subsection{niceguy}

It's not too too bad. A bit about my background before that. I lived in Hong Kong up until EngSci, meaning I took the local curriculum and the public exam (HKDSE). This was really helpful for Calculus I and II. I'm not sure about your year, but in our year, we had this thing called UTEA, which is a free course you can do to learn Mathematics, Physics, Chemistry and Coding. That was my first time learning how to code, and I did fine in both coding courses. Praxis was pain though. I feel like as long as you do assignments as they are given out, instead of doing them on the last possible day, you'll be fine. I was definitely NOT studying 3 nights a week, and I still did fine, apart from Praxis, but there's not much you can do about that. IMO, spending 3 hours versus 1 hour on an essay makes no difference in terms of marks, but the former makes you feel more like a failure, which is a feeling you'll get used to soon.

\section{Notes}

\url{xueqilin.me}
\section{Is IB/AP enough preparation?}

IB/AP will definitely help make everything easier. However, the courses are planned in a way you will understand even if you know basically NOTHING...

\section{Accommodations}

\subsection{Petitions}

This is not an official UofT website telling you "it's okay to submit petitions". Submitting petitions is actually okay. In my year, people were submitting petitions left and right during midterm season for their lab reports. I've seen people with high 80 averages submitting petitions. Your mental health is more important, and have faith in the curve.

Stuff you can petition for via the \href{https://www.google.com/url?sa=t&rct=j&q=&esrc=s&source=web&cd=&cad=rja&uact=8&ved=2ahUKEwjZj-qeh-7-AhX-kokEHX8-C8YQFnoECA0QAQ&url=http\%3A\%2F\%2Fundergrad.engineering.utoronto.ca\%2Fskule-\%2520\%2520\%2520\%2520\%2520\%2520\%2520\%2520\%2520\%2520\%2520\%2520life\%2Fthe-engineering-portal\%2F&usg=AOvVaw145xL8tAHUKBRq6D6rbDhS}{Engineering Portal}: All deliverables (assignments, lab reports, etc) and assessments (quizzes, midterms, finals). It doesn't matter if you're just too busy to prepare for it, or if you're just sick. At this time, you don't need to provide proof of illness. It is rumoured that you get one freebie for sleeping through assessments.

\section{PEY}

You pay for the hopes of getting a job.

\section{More about Courses}

\subsection{UofT Time}

You have probably heard of this before, but lectures, tutorials, and practicals all start 10 minutes after. \textbf{This does not apply to midterms and exams.}

\subsection{Notetaking}

Obviously I am biased. I prefer LaTeX for the more mathy courses, and Markdown for the rest. I know a lot of people take notes on their tablets using apps such as OneNote, Notability, Notion, etc. You can also download and annotate lecture slides directly. A third possibility is to take notes on paper. The advantages are that you never run into technical difficulties, though it might be harder to organise your notes and search for what you need.

\subsection{Summer Courses}

This is a great way to lighten your courseload. Check to see if you can drop courses during the term and take a similar/the same course in the summer. However, you can't do this for every course, e.g. ESC194/ESC195. You will also have to pay extra for summer courses. The nice thing about UofT is that starting second year, you can overload, meaning you can take more courses than necessary, for free! Some 2T4s (whose names shall not be mentioned) were mad enough to take 8 courses in second year. So now nobody else is allowed to (thanks). But you can take 1 extra course, i.e. 7 courses and second year, and in third year and beyond, a maximum of 4.25 credits, or 8.5 courses per semester. Some courses such as seminars count as half-courses. UofT doesn't care about conflicts too, so it is okay to take two courses that have lectures/tutorials at the same time. This is really cool.

% \section{GEARS}
% pls contribute if you've attended/led GEARS, I don't feel it's fair for me to say anything if I haven't been to a single session
