\section{Bahen}

If you don't pronounce it as BAY-hen, you're cringe. This is where the EngSci Common room is located (on the second floor). There are two main lecture halls on the first floor, BA1130 and BA1160, where outlets are located on the first/second row, and at the very back. There are tutorial rooms on the second floor, but there aren't that many outlets there, so you'll have to fight for good spots if you need it. The washrooms switch spots between floors, so don't go into the wrong one. The Wi-Fi here is good until it isn't. Note that CS and ECE students frequent this building, so make sure you're not sitting next to someone who hasn't showered.

\section{Myhal}

On the first floor, there is a big lecture hall called MY150. It supports interactive learning or something, cause students sit around tables and stuff, and each table has a microphone, so you can easily speak to the whole class. There is at least one outlet per seat. You have Praxis lectures here. On the third and fourth floor, you'll find tutorial rooms, with tall tables, which are cool. There are more than enough outlets connected to each table, but you'll have to plug the table to an outlet on the ground, which might not work. This is where you have Praxis studios, and sometimes tutorials for other courses. Finally, there's the Light Fabrication Facility, usually called MyFab, on the fourth floor. You can do 3D printing here and build your prototypes for design teams, Praxis courses, or just for fun! However, you will have to do safety training beforehand to access this lab.

\section{Sandford Fleming}

There's more (smelly) stuff here. In the basement, there's the Pit which occasionally reeks. You can eat here, but spots do be filling up during lunch hours. Otherwise you should be fine. You can get food from Veda and the Hard Hat Cafe located on the same floor. On Friday nights, there is Suds, which is something I'd attend if I had friends. On the first floor, you get the lecture halls SF1101 and SF1105. There are outlets beneath your seats in SF1101 only. There are also two ECF labs (Linux), SF1012 and SF1013 with two printers each. SF1013 is bigger with more computers and a colour printer. On the second floor there's the Engineering and CS library, where you can book study rooms. I haven't done it personally (people only book rooms to study with their friends), but hopefully there are more spots since only Engineers and CS students can book them? It is also (probably) the closest library if you're in an Engineering building.

If you come here on weekends, there is only this one weird door which opens. I won't bother explaining where it is cause I'm bad at that, and they post maps everywhere.

\section{Galbraith}

It is connected to Sandford Fleming inside, which is something you can use strategically in the winter. You get windows ECF labs here, and you also have CIV tutorials here. It sometimes smells. If you're into it, there's also a skate park located conveniently outside.

\section{Libraries}

You can borrow up to 50 books with a 14 day loan period, which auto renews if possible, so you don't pay a fine. In select libraries, you can also find

\begin{itemize}
    \item Windows Laptops/Chromebooks
    \item Tablets
    \item Charging Cables
    \item AV equipment
    \item Video games and Board games
\end{itemize}

Other than that, you can book study rooms in libraries, so you can study and have team meetings with your friends $\in \C\backslash\R$.

\section{Gerstein Science Information Centre}

This is where you study with your friends outside of Engineering, if you have them. It closes pretty early though, at 23:00 on Mondays to Thursdays and 22:00 otherwise.

\section{Robarts Library}

Pronounced ROW-barts. This is where you study with your friends outside of Engineering, if you have them. There is also Robarts Commons which is open 24-hours, and the drama never stops! If you're learning different languages, or if you're Chinese/Japanese/Korean and not illiterate, you can get books written in better languages (not English) in the East Asian library.
