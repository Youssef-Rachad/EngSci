A lot of 2T7s have asked me what they can do to prepare for EngSci. I would say there is really nothing to prepare for; enjoy your summer! But if you insist, here is some stuff you can do (Note: I didn't do most of them and I still did fine)

\section{Health}

Exercise and get 10 hours of sleep every day, because you probably won't get to do that. As for mental health, spend time with your family, friends, and your partner (if any).

\section{I think I am very smart}

Congratulations! If you did high school in North America, your high school grades mean nothing. Also Calc BC means nothing. Getting good grades doesn't mean you're particularly smart (the opposite doesn't have to hold of course). I suggest everyone try out \href{https://byjus.com/jee/jee-advanced-2022-question-paper/https://byjus.com/jee/jee-advanced-2022-question-paper/}{JEE Advanced} (it is actually very fun). If you found it easy, I apologise for being sarcastic.

\section{Academics}

Now there are a lot of things you can do to prepare for EngSci academics. I would suggest you to check out the past papers on \url{courses.skule.ca} to figure out what to prepare for. Depending on how you learn best, pick one (or more) of the below

\begin{itemize}
    \item Textbooks (including LibreTexts!)
    \item Educational Websites, e.g. w3schools, Khan Academy
    \item Online courses (I cannot guarantee how good they are)
    \item Notes from upper years
\end{itemize}

You might want to learn about pointers, or coding in general if you don't have experience. Try learning about limits and $\epsilon-\delta$ proofs, but it's okay if you don't get the gist of it yet. Some basic linear algebra and MATLAB could be useful, but it doesn't really matter as you can learn those in an hour or two anyways.

\section{Projects}

Consider making your own projects! They are fun, and you can put them on your CV. Learn how to use GitHub, and put your cool code there. Or build a thing (anything) and document your design and the building process. It's okay if it doesn't work or if you make mistakes. You learn how to fix problems and develop good practices to avoid them in the first place. It's like how running into bugs teaches you about debugging, testing, and documentation. You might not have as much time for these in the future, so make sure to try them out. I certainly wish I did...

\section{Further Questions}

You might have noticed that I mentioned a lot of stuff without explicitly teaching you how to do it. The answer is "Google it". Some genius answered the same question you have 12 years ago, so I won't bother reinventing the wheel.
