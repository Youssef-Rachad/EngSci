\documentclass[12pt]{article}
\usepackage{../../template}
\usepackage{hyperref}
\author{niceguy}
\title{Lecture 14}
\begin{document}
\maketitle

\section{Numerical Methods}

$$\frac{dy}{dt} = f(t,y)$$
where $f$ is continuously differentiable. Existance and uniqueness theorems give us a unique solution for every $(t_0, y_0)$ pair. \\
Unfortunately, most ODEs cannot be solved analytically, so we must use numerical methods.

\begin{ex}
	Ricatti Equations
	$$y'(x) = q_0(x) + q_0(x)y + q_2(x)y^2$$
	We consider the specific equation
	$$\frac{dy}{dt} = 1 + 2y + ty^2, y(0) = -1$$
	This is technically solvable \url{https://math.stackexchange.com/a/244533}, and you are encouraged to memorise this for coming midterms and finals. However, this can also be done using \textbf{Euler's Method}, which makes use of direction fields (they can be sketched without solving the ODE). The smaller the step size, the more accurate the approximation.
\end{ex}

\subsection{Euler's Method}

We form a partition of $[t_0, T]$ by
$$t_0 < t_1 < t_2 < \dots < t_N = T$$
First, we form an approximation of the solution between $[t_0, t_1]$ using linear interpolation, where
$$y(t_1) = y(t_0) + f(t_0, y_0)(t_1-t_0)$$
Similarly, $y(t_2), y(t_3), \dots, y(t_N) = y(T)$ can also be approximated successively, using the previous approximations. \\
Potential issues
\begin{itemize}
	\item Length of arrows (Does it depend on the ODE?)
	\item Accuracy (Are there ODEs where Euler's Method is just bad?)
\end{itemize}

Euler's method "jumps between" different solution curves, so it is more accurate when solutions converge.
\end{document}
