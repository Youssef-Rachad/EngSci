\documentclass[12pt]{article}
\usepackage{../../template}
\author{niceguy}
\title{Lecture 2}
\begin{document}
\maketitle

\section{Definitions and Terminology}

\begin{defn}
The order of a differential equation is the order of the highest derivative, ordinary or partial, in the equation.
\end{defn}

\begin{ex}
$$ay''' + by'' = f(t)$$
This is a 3$^{\text{rd}}$ order ODE.
\end{ex}

\begin{ex}
$$u_t = \frac{d}{dx}(u_x + y_y + u_z)$$
This is a 1$^{\text{st}}$ order PDE.
\end{ex}

\begin{defn}
$u^{(n)}$ refers to the n$^{\text{th}}$ derivative of $u$.
\end{defn}

The most general equation of an n$^{\text{th}}$ order derivative ODE is:
$$F[t, u, u', u'', \dots, u^{(n)}] = 0$$

\begin{defn}
We say an n$^{\text{th}}$ order ODE is linear if it can be expressed as
$$a_0(t)u(t) + a_1(t)u'(t) + \dots + a_n(t)u^{(n)}(t) = g(t)$$
If $g(t) = 0$ for all $t$, the equation is homogeneous, otherwise it is nonhomogeneous.
\end{defn}

\begin{ex}
Are the following differential equations linear? If so, are they homogeneous or not?

\begin{itemize}
\item $t\sin(t^3) u + 56u''' + t^4 = 0$ \\
\item $t^3u''' + et^2u'' + 4u = 0$ \\
\item $tu + 5\sqrt{u'} + 10\sin(u) = 0$ \\
\end{itemize}
The first equation is linear nonhomogeneous, due to the $t^4$ term. The second equation is linear homogeneous, and the third equation is non linear, due to the second and third terms.
\end{ex}

How could we extend these definitions to PDEs? \\
Consider $$a_0(x,y) u_x + a_1(x,y) u_y + a_2(x,y) u_{xx} + a_3(x,y) u_{xy} + a_4(x,y) u_{yx} + a_5(x,y)u_{yy} + a_6(x,y)u = g(x,y)$$
This is a linear equation. If $g(x,y) = 0 \forall x,y$, it is homogeneous.

\begin{defn}
An ODE is \textit{autonomous} if it does not explicitly depend on the independent variable.
\end{defn}

\begin{ex}
\begin{itemize}
\item $\frac{dy}{dt} = \sin(y) + y^3\ln(y) + e^y$ \\
\item $\frac{dy}{dt} = y'\cos(y) + \frac{y^2}{1 + e^y}$ \\
\end{itemize}
\end{ex}

Does there exist a linear autonomous nonhomogeneous ODE? \\
Consider $$y + y' = 1$$
For it to be nonhomogeneous, a $g(t)$ must exist. It must be independent of $t$, so it can only be a constant (since $g'(t)$ must be 0).

\begin{defn}
A first order ODE is separable if it can be expressed as
$$\frac{du}{dt} = f(u)g(t)$$
\end{defn}

\begin{ex}
Which of the following equations are separable?
\begin{itemize}
\item $\frac{dx}{dt} = t^2 + \frac{\text{arccos}(t)}{\ln(t)}$ \\
\item $u'' = ut^2e^t$ \\
\item $\frac{dv}{dt} = tv + t^2$ \\
\item $\frac{du}{dt} = \frac{1}{2}\left(\sin(u+t) - \sin(u-t)\right) + \sin(t)$ \\
\end{itemize}
Yes, no (second order), no, yes (trig identity).
\end{ex}

\section{Systems of Differential Equations}
This occurs when two or more dependent variables interact with one another.

\begin{ex}
The Lotka-Volterra equations for a predator-prey model. \\
The assumptions are that the prey will only die when eaten, and predators either naturally die or relocate.

$$u_1(t): \text{Population count of prey}$$
$$u_2(t): \text{Population count of predator}$$

Considering the birth and death of prey,

$$\frac{du_1}{dt} = \alpha u_1 - \beta u_1u_2$$

Considering the birth and death of predators,

$$\frac{du_2}{dt} = \gamma u_1u_2 - \delta u_2$$

Intuition tells us that this should be a system of periodic functions, with $u_2$ lagging behind $u_1$.
\end{ex}

\section{Initial Value Problems}

In general, is one initial value enough? \\
No, it depends on the order. E.g. a second order ODE requires 2 initial values.

\begin{ex}
Consider $$\frac{d^2u}{dt^2} = 0$$
Integrating both sides with respect to $t$, we have
$$\frac{du}{dt} = C_0$$
and integrating again gives us
$$u = C_0t + C_1$$
Therefore, two initial values are required to solve for the two constants.
\end{ex}

\begin{defn}
An initial value problem for an n$^{\text{th}}$ order ODE consists of the ODE itself and $n$ initial conditions.
$$u(t_0) = a_0, u'(t_0) = a_1, \dots, u^{(n-1)}(t_0) = a_{n-1}$$
\end{defn}

\end{document}
