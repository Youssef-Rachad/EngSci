\documentclass[12pt]{article}
\usepackage{../../template}
\title{Lecture 24}
\author{niceguy}
\begin{document}
\maketitle

\section{Integral Tansforms}

\begin{ex}
	Suppose you are playing minecraft. Points A and B are separated by 10000 blocks, with a base speed on 4.317 blocks per second. 38.61 minutes of real life is needed. Is there an alternative to save time? \\
	Consider the nether, where moving 1 block there is equivalent to moving 8 blocks on the surface.
\end{ex}

The same can be done with ODEs. We can transform to a new space, where it is easier to solve, then transform back.

\section{Laplace Transform}

\begin{defn}
	Let $f$ be a function defined on $[0,\infty)$. The Laplace transform is defined by

	$$F(s) = \int_0^\infty e^{-st}f(t)dt$$
	The domain of $F$ are the values of $s$ is where the integral is defined and convergent. We also denote $F$ as $\mathcal{L}\{f\}$.
\end{defn}	

It is technically a change of basis.

\begin{ex}
	$$\mathcal{L}\{1\}, t \geq 0$$
	\begin{align*}
		\mathcal{L}\{1\} &= \int_0^\infty e^{-st}dt \\
				 &= -\frac{1}{s}e^{-st} \Big |_0^\infty \\
				 &= \frac{1}{s}
	\end{align*}
	only where $s > 0$.
\end{ex}

\begin{ex}
	$$\mathcal{L}\{e^{at}\}, t\geq 0$$
	\begin{align*}
		\mathcal{L}\{e^{at}\} &= \int_0^\infty e^{-(s-a)t}dt \\
				      &= \frac{1}{s-a}
	\end{align*}
	as above. Again, this only holds where $s-a>0$.
\end{ex}

\begin{ex}
	$$\mathcal{L}\{e^{(a+bi)t}\}, t\geq 0$$
	\begin{align*}
		\mathcal{L}\{e^{(a+bi)t}\} &= \int_0^\infty e^{-(s-a-bi)t} dt \\
					   &= \frac{1}{s-a-bi}
	\end{align*}
	Again, this only holds where $s-a>0$. This is because the complex exponential can be split into the product of the real and imaginary exponential, where the former requires $s-a>0$ to converge to 0 and the latter is bounded and oscillates.
\end{ex}

\begin{thm}
	Linearity of Laplace Transform. \\
	Suppose $\mathcal{L}\{f_1\}$ is defined for $s\geq s_1$ and $\mathcal{L}\{f_2\}$ is defined for $s\geq s_2$. Then
	$$\mathcal{L}\{c_1f_1 + c_2f_2\} = c_1\mathcal{L}\{f_1\} + c_2\mathcal{L}\{f_2\}$$
	defined for $s\geq\text{max}(s_1,s_2)$
\end{thm}

\begin{ex}
	$$\mathcal{L}\{\sin t\}, t\geq 0$$
	\begin{align*}
		\mathcal{L}\{\sin t\} &= \mathcal{L}\left\{\frac{e^{it} - e^{-it}}{2i}\right\} \\
				      &= \frac{1}{2i}\mathcal{L}\left\{e^{it}\right\} - \frac{1}{2i}\mathcal{L}\left\{e^{-it}\right\} \\
				      &= \frac{1}{2i}\left(\frac{1}{s-i} - \frac{1}{s+i}\right) \\
				      &= \frac{1}{s^2+1}
	\end{align*}
\end{ex}

For the Laplace transform to exist on $(a,\infty)$, the integrand has to be integrable and the improper integral has to converge.

\begin{defn}
	A function $f$ is \emph{piecewise continuous} if its domain can be partitioned into intervals where on each interval
	\begin{itemize}
		\item $f$ is continuous on the open subinterval
		\item $f$ has a finite limit at the endpoints of each interval when approached from inside the interval
	\end{itemize}
\end{defn}

Heuristically, $f$ can grow at most exponentially fast.

\begin{defn}
	A function $f(t)$ is of \emph{exponential order} if there exists real constants $M\geq0$, $K>0$, and $a$ such that
	$$|f(t)| \leq Ke^{at} \forall t \geq M$$
\end{defn}

\begin{thm}
	The Laplace transform $\mathcal{L}\{f\}(s)$ exists for $s>a$ if
	\begin{itemize}
		\item $f$ is piecewise continuous on the interval $t \in [0,A]$ for any positive $A$
		\item $f$ is of exponential order
	\end{itemize}
\end{thm}

We can use the comparison test to prove this.
\end{document}
