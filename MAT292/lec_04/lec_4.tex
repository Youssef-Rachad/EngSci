\documentclass[12pt]{article}
\usepackage{../../template}
\author{niceguy}
\title{Lecture 4}

\begin{document}
\maketitle

\section{Review}

\subsection{Solving Separable Equations}

Consider $$\frac{du}{dt} = f(u)g(t)$$
This is solved by
\begin{align*}
	\frac{1}{f(u)} \frac{du}{dt} &= g(t) \\
	\int \frac{1}{f(u)} du &= \int g(t)dt + C, C \in \R \\
\end{align*}

Note: it might not be possible to find an explicit expression for $u$.

\begin{ex}
	$$\frac{du}{dt} = u^2, u(1) = -1$$
	\begin{align*}
		\frac{1}{u^2} \frac{du}{dt} &= 1 \\
		\int \frac{1}{u^2} du &= \int 1 dt \\
		-\frac{1}{u} &= t + C \\
		u &= -\frac{1}{t+C}, C \in \R \\
	\end{align*}
	where $u(1) = -1 \Rightarrow C = 0, u = -\frac{1}{t}$. This is a local solution, as the interval $(-\infty, 0)$ is not included. Consider
	$$f(t) = \begin{cases} -\frac{1}{t} & t \in (0,\infty) \\ -\frac{1}{2+t} & t \in (-\infty, 0] \\ \end{cases}$$
	There is no unique solution for that interval.
\end{ex}

\section{First Order Linear Equations}
$$a_0(t)u(t) + a_1(t)u'(t) = h(t)$$
Assuming $a_1(t) \neq 0$, it can be rearranged as
$$u'(t) + p(t)u(t) = q(t)$$
where $p(t)$ and $q(t)$ are the quotients. \\
Using the product rule, we have
$$\frac{d}{dt} \mu(t)u(t) = \mu'(t)u(t) + \mu(t)u'(t)$$
Factoring $\mu(t)$ out, we have
$$\mu(t)\left(\frac{\mu'(t)}{\mu(t)}u(t) + u'(t)\right)$$
If $\frac{\mu'(t)}{\mu(t)} = p(t)$ and $\mu(t) \neq 0$, this can be solved conveniently. \\
Assuming this is true,
$$\frac{d}{dt} (\mu(t)u(t)) = \mu(t)g(t)$$
Integrating both sides, we can solve for $u$ by
$$u(t) = \frac{\int \mu(t)g(t)dt + C}{\mu(t)}$$
What remains is to solve for $\mu(t)$.
\begin{align*}
	\frac{1}{\mu} \frac{d\mu}{dt} &= p(t) \\
	\int \frac{1}{\mu} d\mu &= \int p(t) dt \\
	\ln|\mu| &= \int p(t) dt \\
	\mu &= \pm e^{\int p(t)dt} \\
\end{align*}
	Note that the $\pm$ term and $C$ terms are constants that are on both sides. Therefore, they can be ignored. In addition, it is an exponential, so it can never be $0$. \\
Hence, defining $\mu(t) = e^{\int p(t)dt}$, the solution is given by
$$u(t) = \frac{1}{\mu(t)} \left(\int_{t_0}^t \mu(s) q(s) ds + C\right)$$
$\mu(t)$ is called the \textbf{integrating factor}.

\begin{ex}
	$$(t^2-1)u' + 2tu = t, u(0) = 1$$
	Putting this into standard form,
	$$u' + \frac{2t}{t^2-1} u = \frac{t}{t^2-1}$$
	The integrating factor is given by
	$$\mu(t) = t^2 - 1$$
	Then
	$$(t^2-1)u(t) = \frac{t^2}{2} + C$$
	And rearranging gives us
	$$u(t) = \frac{t^2}{2(t^2-1)} + \frac{C}{t^2-1}$$
	Substituting the initial conditions,
	$$u(t) = \frac{t^2}{2(t^2-1)} - \frac{1}{t^2-1}$$
\end{ex}
\end{document}
