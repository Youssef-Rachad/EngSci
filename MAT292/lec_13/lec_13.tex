\documentclass[12pt]{article}
\usepackage{../../template}
\author{niceguy}
\title{Lecture 13}
\begin{document}
\maketitle

\section{Review}

\subsection{Repeated Eigenvalues \& Non Diagonalisable}

$$\frac{d\vec{x}}{dt} = A\vec{x}$$
Where $A$ only has one eigenvalue $\lambda$ and eigenvector $\vec{v_1}$. We get our second solution by using the ansatz
$$\vec{\phi_2}(t) = te^{\lambda t}\vec{v_1} + e^{\lambda t}\vec{w}$$
Plugging into the solution (and using the facts $e^t \neq 0$ and $A\vec{v_1} = \lambda\vec{v_1}$) yields the relation
$$(A-\lambda I)\vec{w} = \vec{v_1}$$
We call $\vec{w}$ a \textbf{generalised eigenvector} corresponding to the eigenvalue $\lambda$.

\section{General Techniques}
\begin{enumerate}
	\item Deduce general properties about the solutions of the ODE
	\item Use ansatz to find specific solutions
	\item Arrive at a general solution from these
\end{enumerate}

\section{Generalised Eigenvalues}
There are only 2 possibilities, an "S" or an inverted "S", with arrows pointing outwards in both cases. Consider our previous example
$$\frac{d\vec{x}}{dt} = \begin{pmatrix} 1 & 2 \\ 0 & 1\end{pmatrix}$$
Where the general solution is
$$\vec{\phi}(t) = c_1e^t\begin{pmatrix} 1 \\ 0\end{pmatrix} + c_2\left[te^t\begin{pmatrix} 1 \\ 0\end{pmatrix} + e^t\begin{pmatrix} 0 \\ \frac{1}{2}\end{pmatrix}\right]$$
As $t\rightarrow\infty$, the dominating term becomes $c_2te^t\begin{pmatrix}1 \\ 0\end{pmatrix}$. \\
As $t\rightarrow -\infty$, the function tends to zero. Note that $te^t$ tends to zero.

\begin{ex}
	$$\frac{d\vec{x}}{dt} = \begin{pmatrix} 1 & -1 \\ 1 & 3\end{pmatrix}\vec{x}$$
	The only eigenvalue is $\lambda=2$, and the only eigenvector is given by $\vec{v_1} = \begin{pmatrix} 1 \\ -1\end{pmatrix}$. The generalised eigenvector is given by
	$$\begin{pmatrix} -1 & -1 \\ 1 & 1\end{pmatrix}\begin{pmatrix}w_1 \\ w_2\end{pmatrix} = \begin{pmatrix} 1 \\ -1\end{pmatrix}$$
	A solution is $\vec{w} = \begin{pmatrix} -1 \\ 0\end{pmatrix}$. Finally, all of these can be combined to form the general solution
	$$\vec{\phi}(t) = c_1e^{2t}\begin{pmatrix} 1 \\ -1\end{pmatrix} + c_2\left[te^{2t}\begin{pmatrix} 1 \\ -1\end{pmatrix} + e^{2t}\begin{pmatrix}-1 \\ 0\end{pmatrix}\right]$$
	Trying the derivative at $\begin{pmatrix} 1 \\ 0\end{pmatrix}$ tells us the phase potrait is in the shape on an inverted "S".
\end{ex}

\section{Solving First Order Lienar Systems in General}
If there are two real eigenvalues, the solution is
$$\vec{phi}(t) = c_1e^{\lambda_1 t}\vec{v_1} + c_2e^{\lambda_2 t}\vec{v_2}$$
If there are two complex eigenvalues, the solution is
$$\vec{\phi}(t) = c_1e^{\mu t}\left[\vec{a}\cos(t\nu)-\vec{b}\sin(t\nu)\right] + c_2e^{\mu t}\left[\vec{a}\sin(t\nu)+\vec{b}\cos(t\nu)\right]$$
where $\lambda_1 = \mu+i\nu$ and $\vec{v_1} = \vec{a}+i\vec{b}$. \\
If there is a repeated eigenvalue and two eigenvectors, the solution is given by
$$\vec{\phi}(t) = c_1e^{\lambda t}\vec{v_1} + c_2e^{\lambda t}\vec{v_2}$$
If there is only one eigenvector, the general solution is 
$$\vec{\phi}(t) = c_1e^{\lambda t}\vec{v_1} + c_2\left[te^{\lambda t}\vec{v_1} + e^{\lambda t}\vec{w}\right]$$
where $\vec{w}$ is any vector that satisfies
$$(A-\lambda I)\vec{w} = \vec{v_1}$$

\section{Phase Portraits}
With two distinct real eigenvalues, draw both eigenvectors and see which dominates at different $t$ based on eigenvalues. \\
With two distinct complex eigenvalues, the solution is a spiral (or circle) that goes from or to the origin based on the sign of the real part of the eigenvalues. \\
With a repated eigenvalue but two distinct eigenvectors, the solution is straight lines either from or to the origin, based on the sign of the eigenvalue. \\
If there is only one eigenvector, it is either an "S" or an inverted "S", going from or to the origin based on the sign of the eigenvalue.


\end{document}
