\documentclass[12pt]{article}
\usepackage{../../template}
\author{niceguy}
\title{Lecture 22}
\begin{document}
\maketitle

\section{Finding Particular Solutions}

\begin{tabular}{|c|c|}
	\hline
	$g(t)$ & Ansatz \\
	\hline\hline
	$\alpha e^{kt}$ & $Ae^{kt}$ \\
	\hline
	$a\sin kt$ or $a\cos kt$ & $A\sin(kt) + B\cos(kt)$ \\
	\hline
\end{tabular}

\section{Superposition Principle}

\begin{thm}
	Supposed $y_1(t)$ is a solution to
	$$ay''(t) + by'(t) + cy(t) = g_1(t)$$
	and $y_2(t)$ is a solution to
	$$ay''(t) + by'(t) + cy(t) = g_2(t)$$
	Then $y(t) = y_1(t) + y_2(t)$ is a solution to
	$$ay''(t) + by'(t) + cy(t) = g(t)$$
\end{thm}

The proof is trivial. We then can split the nohomogeneous term.

\begin{ex}
	$$y'' - 3y' - 4y = 3e^{2t} + 2\sin t$$
	We split this into equations
	\begin{equation}\label{a}
		y'' - 3y' - 4y = 3e^{2t}
	\end{equation}
	and
	\begin{equation}\label{b}
		y'' - 3y' - 4y = 2\sin t
	\end{equation}
	Let the particular solution for equation ($n$) be $y_n(t)$, then the particular solution for the equation is
	$$y(t) = y_1(t) + y_2(t)$$
\end{ex}

\begin{ex}
	$$y'' - 3y' - 4y = 2e^{-t}$$
	If we try the particular solution $y_p(t) = Ae^{-t}$, we get
	\begin{align*}
		Ae^{-t} + 3Ae^{-t} - 4Ae^{-t} &= 2e^{-t} \\
		0 &= 2e^{-t}
	\end{align*}
	which is bad. One can try $y(t) = Ate^{-t}$ by consulting the solution manual. Therefore, a different ansatz is needed if it happens to be the general solution.
\end{ex}

\section{Variation of Parameters}
The method of undetermined coefficients is not a general approach, as ansatz are needed for every case, yet not every case is covered, and ans\"atze do not work if they happen to be the general solution. \\
Now consider
$$\frac{d\vec{x}}{dt} = P(x)\vec{x}(t) + \vec{g}(t)$$
and assume we have linearly independent vectors $\vec{x_1}(t), \vec{x_2}(t)$ that are solutions to the homogeneous equation. Then our guess of a particular funtion is
$$\vec{x}(t) = u_1(t)\vec{x_1}(t) + u_2(t)\vec{x_2}(t) = X(t)\vec{u}(t)$$
This is similar to the general solution for the homogeneous equation, but only with functions instead of constants as coefficients. Plugging into the ODE,
\begin{align*}
	\frac{d\vec{x}}{t} &= P(t)\vec{x}(t) + \vec{g}(t) \\
	X'(t)\vec{u}(t) + X(t)\vec{u}'(t) &= P(t)X(t)\vec{u}(t) + \vec{g}(t) \\
	P(t)X(t)\vec{u}(t) + X(t)\vec{u}'(t) &= P(t)X(t)\vec{u}(t) + \vec{g}(t) \\
	X(t)\vec{u}'(t) &= \vec{g}(t) \\
	\vec{u}'(t) &= X^{-1}(t)\vec{g}(t)
\end{align*}
Note that on the third line, we substitute the first term back into the homogeneous equation, and on the last line, $X(t)$ is invertible as its determiant is nonzero $\forall t$. Integrating, we then have
$$\vec{u}(t) = \vec{c} + \int X^{-1}(t)\vec{g}(t)dt$$
and then
$$\vec{x}(t) = X(t)\vec{c} + X(t)\int X^{-1}(t)\vec{g}(t)dt$$
The first term is just the general solution, so the particular solution can be reduced to the second term.

\begin{ex}
	$$\frac{d\vec{x}}{t} = \begin{bmatrix} 1 & -4 \\ 2 & -5\end{bmatrix}\vec{x}(t) + \begin{bmatrix} 10\cos t \\ 2e^{-t}\end{bmatrix}$$
	The general solution is spanned by
	$$\vec{x}_1(t) = e^{-t}\begin{bmatrix} 2 \\ 1\end{bmatrix}$$
	and
	$$\vec{x}_2(t) = e^{-3t}\begin{bmatrix} 1 \\ 1\end{bmatrix}$$
	The inverse matrix is
	$$X^{-1}(t) = \begin{bmatrix} e^t & -e^t \\ -e^{3t} & 2e^{3t}\end{bmatrix}$$
	So
	\begin{align*}
		\vec{x}_p(t) &= X(t)\int X^{-1}(t)\vec{g}(t)dt \\
			     &= X(t)\int\begin{bmatrix} e^t & -e^t \\ -e^{3t} & 2e^{3t}\end{bmatrix}\begin{bmatrix} 10\cos t \\ 2e^{-t}\end{bmatrix}dt \\
			     &= \begin{bmatrix}7\cos t + 9\sin t + 2(1-2t)e^{-t} \\ 2\cos t + 4\sin t + 2(1-t)e^{-t}\end{bmatrix}
	\end{align*}
	Note that some steps were skipped.
\end{ex}

This can be applied to second order ODEs, as they are but special cases of linear first order ODEs. Then for
$$ay''(t) + by'(t) + cy(t) = g(t)$$
and solutions $y_1(t), y_2(t)$,
$$X(t) = \begin{bmatrix} y_1 & y_2 \\ y_1' & y_2'\end{bmatrix}$$
\end{document}
