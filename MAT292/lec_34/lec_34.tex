\documentclass[12pt]{article}
\usepackage{../../template}
\author{niceguy}
\title{Lecture 34}
\begin{document}
\maketitle

\section{1-D Heat Equation}

From the last lecture, we assumed $\lambda > 0$. This does not work in general. However, no solution is found if $\lambda \leq 0$, so it can be discarded. Substituting into the equation for $X$ yields

$$X(\sqrt{\lambda}x) = 0$$
and
$$\lambda = \frac{n^2\pi^2}{L^2}$$

\begin{align*}
	T'(t) + \alpha^2\lambda T(t) &= 0 \\
	T'(t) &= -\alpha^2\lambda T(t) \\
	T(t) &= Ce^{-\alpha^2\lambda t} \\
	     &= Ce^{-\frac{\alpha^2n^2\pi^2}{L^2}t}
\end{align*}

The solution is then their product

$$u_n(x,t) = c_n\sin\left(\frac{n\pi x}{L}\right)e^{-\frac{\alpha^2n^2\pi^2}{L^2}t}$$

Using the initial condition
$$u(x,0) = f(x)$$
we have
$$f(x) = c_n\sin\left(\frac{n\pi x}{L}\right)$$
However, this "fixes" our initial condition, which should not be possible.

\begin{thm}
	If $u_1(x,t)$ and $u_2(x,t)$ solve
	$$\begin{cases} u_t(x,t)=\alpha^2u_{xx}(x,t) & \\ u(0,t)=0 & u(L,t)=0\end{cases}$$
		then $c_1u_1(x,t) + c_2u_2(x,t)$ also solves the above equation for any coefficients $c_1,c_2\in\R$.
\end{thm}

Then
$$f(x) = \sum_{n=1}^\infty c_n\sin\left(\frac{n\pi x}{L}\right)$$

$f$ is odd, but it does not matter, as the solution is only defined on $x\in[0,L]$. The coefficients $c_n$ can easily be found, given an intial condition, which yields the solution.

\section{Final Review}

Consider the autonomous ODE
$$y'(t) = y^2(y-5)(y+10)$$
The equilibrium solutions are 0, 5, -10. Given the initial value $y(0)=1$,
$$\lim_{t\rightarrow\infty}y(t) = 0$$
and
$$\lim_{t\rightarrow -\infty}y(t) = 5$$
\end{document}
