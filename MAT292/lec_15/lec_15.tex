\documentclass[12pt]{article}
\usepackage{../../template}
\title{Lecture 15}
\author{niceguy}
\begin{document}
\maketitle

\section{Euler's Method}
If we partition $[t_0,T]$ such that the distance between adjacent points are constant, we can let that distance be $h = t_{n+1} - t_n$, and approximations are given by
$$y_{n+1} = y_n + hf(t_n,y_n)$$

\subsection{Euler's Method as an Integral Approximation}
\begin{align*}
	y'(t) &= f(y,t) \\
	\int_{t_n}^{t_{n+1}} y'(t) dt &= \int_{t_n}^{t_{n+1}} f(y,t) dt \\
	y(t_{n+1}) - y(t_n) &= \int_{t_n}^{t_{n+1}} f(y,t) dt \\
	y(t_{n+1}) &\approx y(t_n) + hf(y_n,t_n)
\end{align*}
where we use the approximation
$$f(y,t) \approx f(y_n,t_n)$$
in the range $[t_n,t_{n+1}]$. More formally, we define
$$y_{n+1} = y_n + hf(y_n,t_n)$$
and use the approximation
$$y(t_{n+1}) \approx y_{n+1}$$

\subsection{Euler's Method as a Taylor Approximation}
Assuing $y$ has a Taylor series, we can approximate $y$ using its first order Taylor polynomial
$$y(t) \approx y(t_n) + y'(t_n)(t-t_n)$$
which is equivelant to Euler's Method.

\subsection{Improving Euler's method}
If we treat it as a forward difference quotient, this can be improved by taking a better approximation of the derivative, which gives us the \textbf{Runge-Kutta method}. \\
If we treat it as an integral approximation, we can use better integral approximations, which gives us the \textbf{Improved Euler Method}. \\
If we treat it as a Taylor Polynomial, we can improve it by taking more terms, which gives us the \textbf{Power Series Solution}.

\section{Sources of Errors}

\subsection{Global Truncation Error}
$$E_n = y(t_n) - y_n$$
The error stacks, as we use $y_n$ and not $y(t_n)$ for our next approximation.

\subsection{Local Truncation Error}
When calculating $e_{n+1}$, we use $y(t_n)$ instead of $y_n$ to calculate the error. This is the error due to linear approximation.

\subsection{Error Bounding}
Taylor's Remainder Theorem is
$$y(t) = y(t_n) + y'(t_n)(t-t_n) + \frac{y''(\xi)}{2}(t-t_n)^2$$
where $t \in [t_n,t_n+h]$. To calculate for $e_{n+1}$, we substitute $t = t_{n+1}$ to get
\begin{align*}
	y(t_{n+1}) &= y(t_n) + y'(t_n)(t_{n+1}-t_n) + \frac{y''(\xi)}{2}(t_{n+1}-t_n)^2 \\
	y(t_{n+1}) &= y_{t_{n+1}} + \frac{y''(\xi)}{2}h^2 \\
	|e_{n+1}| &= \frac{y''(\xi)}{2}h^2 \\
		  &= \frac{M}{2}h^2
\end{align*}
where $M$ is chosen such that
$$|y''(t)| \leq M \forall t\in[t_n,t_{n+1}]$$
\begin{align*}
	y' &= f(t,y) \\
	y'' &= f_t(t,y(t)) + f_y(t,y(t))\times y'(t) \\
	    &= f_t(t,y) + f_y(t,y)f(t,y) \\
\end{align*}
What remains is to bound this expression. \\
Our assumptions are that $y$ is twice continuously differentiable, and $f_t, f_y, f$ are continuous functions. \\
Even if we don't have access to the solution, a bound may still be obtained.

\begin{ex}
	$$y'(t) = \arctan(y) + e^{-t}, y(0) = 1$$
	for $t\in[0,4]$. Then
	\begin{align*}
		f &= \arctan(y) + e^{-t} \\
		f_t &= -e^{-t} \\
		f_y &= \frac{1}{1+y^2}
	\end{align*}
	As the arctan function is bounded, and $e^{-t}$ is obviously bounded in the region, $f$ is bounded, $f_t$ is bounded, and $f_y$ is bounded (it is always smaller than or equal to 1). Therefore $\exists$ an upper bound $M$.
\end{ex}
\end{document}
