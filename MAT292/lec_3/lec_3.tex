\documentclass[12pt]{article}
\usepackage{../../template}
\title{Lecture 3}
\author{niceguy}
\begin{document}
\maketitle
\section{Questions from last lecture}

\subsection{What if integral curves intersect?}

If we lack uniqueness, the IVP is ill-posed, usually because the model is incorrect.

\begin{ex}
$$\frac{du}{dt} = u^{\frac{2}{3}}, u(0) = 0$$
Both $u(t) = 0$ and $u(t) = \left(\frac{t}{3}\right)^3$ are solutions.
\end{ex}

\subsection{Initial values are imposed on higher derivatives instead of the dependent variable}

$$u'' = -u$$
The solution is in the form of
$$u = C_0\sin(t) + C_1\cos(t)$$
If initial conditions are imposed on $u$, there may not be a solution, e.g.
$$\begin{cases} u(0) = 0 & \\ u(\pi) = 1 & \\ \end{cases}$$

\section{Global and Local solutions}

\begin{ex}
$$\frac{du}{dt} = 1 + u^2, u(0) = 0$$
We know that $\tan(t)$ is a solution, but it cannot be global, as it diverges at $\pm \frac{\pi}{2}$.
\end{ex}

\section{Direction (Slope) Fields}
$$\frac{du}{dt} = f(t,u)$$
$f$ is the slope of $u$. By plotting a $t-u$ plane, and drawing out the slopes $f$, we get adirection field of the ODE. In fact, if the equation is autonomous, the plot is independent of $t$, which simplifies it.

\subsection{Slope Fields vs Integral Curves}
You do not need to solve for the general solution when drawing slope fields. However, they are less "accurate", as they are only a first order Taylor approximation (cooler way of saying "tangent").

\section{Equilibrium}

\begin{defn}
Consider the first order ODE $$\frac{du}{dt} = f(u)$$
Equilibrium solutions are those satisfying $$f(u) = 0$$
Stable equilibria are when solutions "near" it tend towards them. \\
Unstable equilibria are when solutions "near" it tend away from them.
\end{defn}

\begin{ex}
$$\frac{du}{dt} = \cos(u)$$
Then at $u = \frac{\pi}{2}$, solutions slightly below will increase (positive slope) and those above will decrease (negative slope), so it is a stable equilibrium. Conversely, $u = \frac{3\pi}{2}$ is an unstable equilibrium.
\end{ex}

There are asymptomatically stable equilibrium points, unstable equilibrium points, and two types of semistable equilibrium points, where solutions tend to increase/decrease (type i and ii) when they are "near" the equilibrium.

\begin{ex}
An ODE with a semistable equilibrium point (type i): $$\frac{du}{dt} = u^2$$
An ODE with a semistable equilibrium point (type ii): $$\frac{du}{dt} = -u^2$$
An ODE with infinitely many semistable equilibrium points (type i): $$\frac{du}{dt} = \sin(u) + 1$$
An ODE with infinitely many semistable equilibrium points (type ii): $$\frac{du}{dt} = \sin(u) - 1$$
\end{ex}
\end{document}
