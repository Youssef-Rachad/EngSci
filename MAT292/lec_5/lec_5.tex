\documentclass[12pt]{article}
\usepackage{../../template}
\author{niceguy}
\title{Lecture 5}

\begin{document}
\maketitle

\section{Review}
Solving a first order ODE
\begin{enumerate}
	\item Put it into standarm form \\
	\item Integrate $p(t)$ \\
	\item The integrating factor is the exponential of the integral of $p(t)$ \\
	\item Plug in the formula \\
\end{enumerate}

\textbf{Why does the +C term not matter?} \\
When taking the exponential, it results in a nonzero constant coefficient, which is cancelled out on both sides of the equation.

\begin{ex}
	$$y' + 2ty = t$$
	This is in the right form, so the integrating factor could be immediately found as $e^{t^2}$.
	\begin{align*}
		ye^{t^2} &= \int te^{t^2} dt \\
		ye^{t^2} &= \frac{1}{2} e^{t^2} + C \\
		y &= Ce^{-t^2} + \frac{1}{2}, C \in \R \\
	\end{align*}
\end{ex}

\section{Mathematical Modelling: Rocket Ship}
Assumptions: air resistance is negligible. \\
We start with the equations
$$F = ma$$
and
$$w(x) = -\frac{mgR^2}{(R+x)^2}$$
Where $w(x)$ is the gravitational force at height $x$, and $R$ is Earth's radius. \\
Equating both sides,
$$m\frac{dv}{dt} = -\frac{mgR^2}{(R+x)^2}$$
There are too many variables! \\
Using chain rule, we can rewrite
$$\frac{dv}{dt} = \frac{dv}{dx} \frac{dx}{dt} = v \frac{dv}{dx}$$
So
$$mv \frac{dv}{dx} = -\frac{mgR^2}{(R+x)^2}$$
This is separable, so
\begin{align*}
	v \frac{dv}{dx} &= -\frac{gR^2}{(R+x)^2} \\
	\int vdv &= -gR^2 \int \frac{dx}{(R+x)^2} \\
	\frac{1}{2} v^2 &= \frac{gR^2}{R+x} + C \\
\end{align*}
The initial condition of $v(0) = v_0$ gives us the constant $C = \frac{1}{2} v_0^2 - gR$. The solution is then by
$$v = \pm \sqrt{\frac{2gR^2}{R+x} + v_0^2 - 2gR}$$
\subsection{Maximum Altitude}
The maximum altitude is when $v = 0$, or
\begin{align*}
	\frac{2gR^2}{R+x} + v_0^2 - 2gR &= 0 \\
	2gR^2 + (R+x)(v_0^2-2gR) &= 0 \\
	R+x &= -\frac{2gR^2}{v_0^2-2gR} \\
	R+x &= \frac{2gR^2}{2gR-v_0^2} \\
	x &= \frac{Rv_0^2}{2gR-v_0^2} \\
\end{align*}

\subsection{Required initial speed to reach a maximal height of $\xi$.}
\begin{align*}
	\frac{2gR^2}{R+\xi} + v_0^2 - 2gR &= 0 \\
	v_0 &= \pm \sqrt{2gR - \frac{2gR^2}{R+\xi}} \\
	    &= \pm \sqrt{\frac{2gR\xi}{R+\xi}} \\
\end{align*}

\subsection{Exit Velocity}
Say we are trying to determine the speed needed to never return to EngSci. We then want to reach a maximal height of "$\infty$", or more formally, as $\xi \rightarrow \infty$. We get
$$v_0 = \sqrt{2gR}$$

\section{Existance and Uniqueness}

\begin{thm}
	Consider the first order IVP
	$$\begin{cases} u'+p(t)u = g(t) & \\ u(t_0) = u_0 & \\ \end{cases}$$
	and an open interval $I = (\alpha, \beta)$. If $t_0 \in I$, $g(t)$ and $p(t)$ are both continuous on $I$, then this IVP has a unique solution defined on $I$.
\end{thm}

Proof Sketch: \\
Let $F(t)$ be the antiderivative of $p(t)$. Since $p$ is continuous, $F(t)$ is continuous and differentiable over $I$. Letting $\mu(t) = e^{F(t)}$, $\mu$ is also defined and continous over $I$, so the following equation holds $\forall t \in I$.
$$\frac{d}{dt}(\mu(t)u(t)) = \mu(t)g(t)$$
Since $\mu$ and $g$ are continous, their product is continuous and hence integrable. So we can integrate both sides to get $\mu(t)u(t) = G(t) + C$ where $C \in \R$ and $G(t)$ is the antiderivative of the right hand side. Since $\mu \neq 0$ (it is an exponential, we can conclude that
$$u(t) = \frac{G(t) + C}{\mu(t)}$$

\begin{thm}
	Consider the first-order IVP:
	$$\begin{cases} u'=f(t,u) & \\ u(t_0) = u_0 & \\ \end{cases}$$
	and the open rectangle $(\alpha, \beta) \times (\gamma, \delta)$. If $(t_0, u_0) \in (\alpha, \beta) \times (\gamma, \delta)$, $f$ is continuous on the rectangle, and $f_u$ is continuous on the rectangle, then this IVP has a unique solution defined on some subinterval $(t_0-h, t_0+h) \subset (\alpha, \beta)$.
\end{thm}
\end{document}
